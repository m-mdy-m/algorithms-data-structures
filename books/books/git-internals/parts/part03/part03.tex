\part{Data Model: The Heart of \git}
\label{part:data-model}

\begin{partintro}
\lettrine[lines=3]{T}{his is where} \git\ gets interesting. The data model is \git's core insight—the idea that makes everything else possible. This part develops the mathematical foundations of content-addressable storage, directed acyclic graphs, and object representations.

\vspace{1em}
\textbf{What You'll Learn:}
\begin{itemize}[noitemsep]
    \item \textbf{Content-Addressable Storage:} Why hashing content is revolutionary
    \item \textbf{The Four Object Types:} Blob, tree, commit, tag—structure and purpose
    \item \textbf{DAG Theory:} Formal graph-theoretic foundations
    \item \textbf{SHA-1 Mathematics:} Collision probability and cryptographic properties
    \item \textbf{Object Storage:} How objects are stored and retrieved efficiently
\end{itemize}

\begin{quote}
\textit{``Show me your flowcharts and conceal your tables, and I shall continue to be mystified. Show me your tables, and I won't usually need your flowcharts; they'll be obvious.''}
\hfill--- \textsc{Fred Brooks}
\end{quote}
\end{partintro}

\chapter{Content-Addressable Storage}
\chapter{The Four Object Types}
\chapter{The DAG: Mathematical Foundations}
\chapter{SHA-1: Cryptographic Hash Functions}
\chapter{Object Storage and Retrieval}
\chapter{Pack Files and Delta Compression}
\chapter{The .git Directory: Repository Anatomy}
