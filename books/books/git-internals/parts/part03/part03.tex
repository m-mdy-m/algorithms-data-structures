\part{The Data Model}
\label{part:data-model}

\begin{partintro}
Git's core insight. Everything builds on this. Content-addressable storage. Four object types. The DAG.
\end{partintro}

\chapter{Content-Addressable Storage}
% Hash as identity
% SHA-1 and SHA-256
% Why this solves problems
% Automatic deduplication
% Integrity verification

\chapter{Object Types}
% Blob: file content
% Tree: directory structure
% Commit: snapshot
% Tag: annotated reference
% How they connect
% Object format

\chapter{The DAG}
% Why graph not tree
% Parent relationships
% Merge commits
% Graph properties
% Implications for algorithms

\chapter{References}
% Branches as pointers
% Tags
% HEAD and symbolic refs
% Remote tracking
% Packed refs
% Reference transactions

\chapter{The Index}
% What it is and why
% Binary format
% Stat cache
% Multiple stages (merge state)
% Extensions (cache-tree, split-index)
% Three-way merge foundation

\chapter{Repository Structure}
% .git/ directory layout
% objects/ and refs/
% hooks/, info/, logs/
% config and HEAD
% What lives where