\part{The Data Model: Git's Core Insight}
\label{part:data-model}

\begin{partintro}
This is what makes git work. The data model is elegant, simple, and powerful.

Everything else in git follows from this model.
\end{partintro}

\chapter{Content-Addressable Storage}
% Hash everything
% SHA-1 (and SHA-256)
% Why hashing solves problems
% Deduplication for free

\chapter{The Four Object Types}
% Blob: file content
% Tree: directory structure
% Commit: snapshot + metadata
% Tag: named reference
% How they relate

\chapter{The DAG: Directed Acyclic Graph}
% Commits form a DAG
% Why DAG not tree?
% Merge commits have multiple parents
% Mathematical properties

\chapter{References: Branches and Tags}
% refs/heads/, refs/tags/, refs/remotes/
% Branches are just pointers
% HEAD is special
% Symbolic refs

\chapter{The Index (Staging Area)}
% What it is and why
% Three-way merge foundation
% How add/rm modify it
% Index file format

\chapter{The Working Directory}
% Tracked vs untracked
% Modified vs staged vs committed
% .git/ directory structure
% What's stored where