\part{Design Patterns in Git's Implementation}
\label{part:patterns}

\begin{partintro}
Git's codebase uses recognizable design patterns. Not always explicitly, but they're there.

Identifying these patterns helps you understand the code and teaches you about software architecture.
\end{partintro}

\chapter{Immutable Data Structures}
% Objects never change
% Benefits: caching, sharing, threading
% How git exploits immutability

\chapter{Content-Addressable Store Pattern}
% Hash as identity
% Beyond git: IPFS, Nix, Docker
% When to use this pattern

\chapter{Command Pattern}
% How builtin commands work
% Registration and dispatch
% Extensibility through this pattern

\chapter{Strategy Pattern}
% Merge strategies
% Diff algorithms
% Pluggable implementations

\chapter{Builder Pattern}
% Tree and commit construction
% Index building
% Why builders matter here

\chapter{Object Pool and Caching}
% Parsed object cache
% Pack file caches
% Memory management patterns

\chapter{Plugin Architecture}
% Hooks as extension points
% Custom merge drivers
% Credential helpers
% How git stays extensible