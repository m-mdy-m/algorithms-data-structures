\part{The Object Model}
\label{part:object-model}

\begin{partintro}
Git's core insight. Content-addressable storage. Four object types. How they relate. Why this model is powerful. This is the heart of everything.
\end{partintro}

\chapter{Content-Addressable Storage}
% Hash as identity
% SHA-1 mechanics
% SHA-256 transition
% Why hashing solves problems
% Deduplication
% Integrity verification
% Hash collision handling

\chapter{Object Types and Structure}
% Blob: raw file content
% Tree: directory structure
% Commit: snapshot + metadata
% Tag: named references
% Object header format
% Size encoding
% Type identification

\chapter{Object Serialization}
% Binary format
% Zlib compression
% Header structure
% Content encoding
% Parsing implementation
% Writing implementation

\chapter{Loose Objects}
% File system layout (.git/objects)
% Fanout directories
% Filename from hash
% Permissions and security
% When created
% Storage inefficiency

\chapter{References System}
% refs/heads/ - branches
% refs/tags/ - tags
% refs/remotes/ - remote tracking
% HEAD and symbolic refs
% Packed refs format
% Reference transactions
% Atomicity guarantees

\chapter{The Index (Staging Area)}
% What it stores
% Binary format
% Stat information
% Flags and extended flags
% Multiple stages (merge state)
% Cache tree extension
% Split index
% FS monitor integration
% Why three-way merge needs this

\chapter{RefLog}
% What reflog tracks
% Format and storage
% Expiration policy
% Recovery using reflog
% Per-reference logs
% Use in garbage collection