\part{Security and Integrity}
\label{part:security}

\begin{partintro}
Git's "trust but verify" model. Cryptographic hashing. Attack surfaces. How Git protects data integrity and handles malicious input.
\end{partintro}

\chapter{Cryptographic Hashing}
% SHA-1 usage
% Collision attacks
% SHA-256 transition
% Hardened SHA-1
% Hash verification everywhere
% Performance cost

\chapter{Object Integrity}
% Every object is checksummed
% Corruption detection
% Fsck implementation
% Repair strategies
% Data recovery

\chapter{Input Validation}
% Parsing untrusted input
% Path validation
% Size limits
% Recursion limits
% Injection prevention

\chapter{GPG Integration}
% Signed commits
% Signed tags
% Verification process
% Key management
% Trust models

\chapter{Attack Surfaces}
% Malicious repositories
% Symbolic link attacks
% Path traversal
% Submodule risks
% Hook execution
% Mitigations

\chapter{Fuzzing and Testing}
% Fuzz testing
% Edge cases
% Malformed input
% Stress testing
% Regression prevention