\chapter*{Preface}
\addcontentsline{toc}{chapter}{Preface}

\lettrine{I}{  started reading} git's source code because I was bored. 

Not in a bad way. I'd just finished a project and didn't want to start another one yet. So I cloned git, opened \file{common-main.c}, and started reading.

That was six months ago.

What started as "let me see how this works" turned into something else. I kept finding things that made me stop and think. Not just "oh that's clever" but "wait, why did they do it THIS way and not that way?"

Like why is everything a hash? Why DAG instead of a tree? Why are branches basically just text files with a hash in them? Why does merge work the way it does?

These aren't git usage questions. These are software engineering questions.

\section*{What This Actually Is}

This isn't a git tutorial. There are plenty of those already. Pro Git is excellent if you want to learn how to use git.

This is about how git is built. The code. The data structures. The algorithms. The design decisions. The patterns. The trade-offs.

I'm reading the source code and explaining what I find. Not line by line of everything—that would be insane and boring. But the interesting parts. The parts that teach you something about building software.

Why does git's source look like this? Why these data structures? What problems were they solving? What did they try that didn't work? Where are the clever bits?

\section*{Who This Is For}

You if you've ever wondered "how does git actually work under the hood?"

You if you want to see what a mature, complex C codebase looks like.

You if you're interested in software architecture, not just using tools.

You if you want to understand the engineering, not just memorize commands.

You don't need to be a C expert. You don't need to be a git expert. You just need to be curious about how things work.

\section*{What I'm Not Doing}

I'm not teaching you git commands. I assume you already know basic git or can learn it elsewhere.

I'm not proving theorems. This isn't a CS paper. When I talk about complexity or algorithms, I keep it practical.

I'm not being comprehensive. Git has thousands of files. I'm focusing on the core stuff that matters.

I'm not being perfect. I'll probably get some things wrong. That's fine. The goal is understanding, not perfection.

\section*{How I'm Writing This}

I'm trying to write the way I actually think about code. Informal. Direct. Sometimes messy.

When I say "this is clever" I mean it. When I say "I don't get why they did this" I mean that too.

I'll show you code. Real code from git. Not cleaned up pseudocode. The actual thing.

I'll explain design patterns when I see them. Not because I want to sound academic, but because patterns matter. They're how you make sense of large codebases.

\section*{The Structure}

Part I: What git is, why it exists, basic concepts. Quick. Just enough context.

Part II: The architecture. How the codebase is organized. Where things are. How to navigate it.

Part III: The data model. This is the core insight that makes git work. Objects, hashes, DAG.

Part IV: Storage and performance. Pack files, delta compression, why git is fast.

Part V: Design patterns. The engineering patterns I see throughout the code.

Part VI: Core algorithms. Merge, diff, tree traversal. The complicated stuff.

Part VII: Extension points. Hooks, custom commands, how git lets you extend it.

Part VIII: Comparisons. Git vs Mercurial vs SVN. What's different and why.

\section*{A Note on the Code}

Git is written in C. Old school C. Pointers everywhere. Manual memory management. Platform-specific macros.

If that scares you, don't worry. I'll explain the tricky bits. And honestly, reading real C code is a useful skill even if you never write C.

The patterns and ideas in git's code apply to any language. Content-addressable storage works in Python. DAG algorithms work in Go. Design patterns are language-agnostic.

\section*{This Is Incomplete}

I'm still learning this stuff. Still reading the code. Still finding new things.

So this book is a snapshot. What I understand right now. In six months I'll probably understand more and want to rewrite parts of it.

That's okay. That's how learning works.

\section*{Why Bother?}

Because git is everywhere. Every developer uses it. Most don't understand it.

Because understanding tools makes you better with them.

Because git's codebase has lessons about software engineering that apply way beyond version control.

Because it's interesting.

Let's start.

\vfill
\begin{flushright}
\textit{Mahdi}\\
\textit{2025}
\end{flushright}

\clearpage