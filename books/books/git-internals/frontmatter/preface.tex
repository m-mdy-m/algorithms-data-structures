\chapter*{Preface}
\addcontentsline{toc}{chapter}{Preface}

\lettrine{I}{  had just finished} a project and wanted to start something new. I was working on gix—a Git wrapper I'd been thinking about—and realized I didn't actually understand Git well enough.\\
Sure, I could use it. I knew the commands. But how does it work? Why is it designed this way? What makes it different from other version control systems?\\
I read some articles. Looked at Pro Git's internals chapter. Found a small booklet on Git internals. But I wanted more. I wanted to see the actual code. The decisions. The trade-offs.\\
So I cloned the Git repository and started reading.

\section*{Why I'm Writing This}

At first, I was just making notes for myself. Trying to understand the codebase. Mapping out how things connect.\\
Then I thought maybe I could write an article about it. Share what I learned.\\
But the more I read, the more I found. This wasn't article material. There was too much to say.\\
So here we are. A book about Git's internals from an engineering perspective.\\

\section*{What This Book Actually Is}
This is me reading Git's source code and explaining what I find.\\
Not line by line—that would be boring and pointless. But the important parts. The clever solutions. The patterns that keep showing up. The reasons things are built the way they are.\\
I care about software engineering. How systems are designed. Why certain approaches work. What problems they solve. Git is interesting because it does things differently and those differences matter.

\section*{Who This Is For}

You if you've used Git but want to understand how it works underneath.\\
You if you're interested in system design and want to see a real, mature codebase.\\
You if you like reading code and learning from it.\\
You don't need to be a C expert. I'll explain the tricky bits. You don't need to know Git internals already. That's what this book is about.

\section*{What This Isn't}

This isn't a Git tutorial. I'm not teaching you how to use Git. There are better resources for that.\\
This isn't comprehensive. Git's codebase is massive. I'm focusing on the core ideas and the interesting engineering choices.\\
This isn't perfect. I'm still learning this stuff. I'll probably get some things wrong. That's okay.

\section*{How I'm Approaching This}

I started with Git's first commit. The original implementation by Linus Torvalds. It's small and shows the core model clearly.\\
Then I worked forward, looking at how the codebase evolved. What got added. What got changed. Why.\\
I'm trying to explain not just what the code does but why it's designed that way. What problem was being solved? What alternatives existed? What trade-offs were made?

\section*{This Is Ongoing}

I'm still reading the code. Still learning. This book reflects what I understand right now.\\
Maybe in six months I'll understand more and want to revise some sections. That's how learning works.\\
Consider this a snapshot of the journey, not the final destination.

\section*{Why Bother?}

Because Git is everywhere and most people don't understand it.\\
Because understanding your tools makes you better at using them.\\
Because Git's design has lessons that apply beyond version control.\\
Because it's interesting.\\
Let's start.

\vfill
\begin{flushright}
	\textit{Mahdi}\\
	\textit{2025}
\end{flushright}

\clearpage