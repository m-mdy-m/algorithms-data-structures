\chapter*{Preface}
\addcontentsline{toc}{chapter}{Preface}

\lettrine{M}{athematics is not learned}—it is lived. This book emerged not from a plan, but from a necessity I could no longer ignore.

During my work on \textit{Arliz} and \textit{The Art of Algorithmic Analysis}, I confronted an uncomfortable truth: my mathematical foundation was insufficient. Not superficially—I could manipulate symbols, apply formulas, solve standard problems—but fundamentally. I lacked the deep, intuitive understanding that transforms mathematics from a tool into a language of thought.\\
The realization was humbling. Here I was, attempting to write comprehensive treatments of data structures and algorithmic analysis, yet stumbling over concepts that should have been second nature. When working through recurrence relations, I found myself mechanically applying methods without truly grasping why they worked. When analyzing probabilistic algorithms, I could follow the calculations but couldn't see the underlying structure. When dealing with matrix operations in multidimensional arrays, the algebra felt arbitrary rather than inevitable.\\
This gap became impossible to ignore.

\section*{The Decision to Begin Again}

I made a choice: to pause my other work and return to the beginning. Not to the beginning of computer science, but to the beginning of mathematical thought itself. If I was to write honestly about computation, I needed to understand the mathematics that makes computation possible—not as a collection of techniques, but as a coherent intellectual tradition.\\
I began reading widely. Aristotle's \textit{Organon} for logical foundations. Al-Khwarizmi's \textit{Al-Jabr wa-l-Muqabala} to understand algebra's origins. Ibn Sina's \textit{Al-Shifa} for its systematic treatment of mathematics within broader philosophical context. Euclid's \textit{Elements} to see how axiomatic thinking crystallized geometric intuition. The works of Descartes, Leibniz, Euler, Gauss—each revealing how mathematical structures emerged from intellectual necessity.\\
What struck me most was the continuity. These were not isolated discoveries but conversations across centuries. Khwarizmi built on Greek algebra, which drew from Babylonian methods. Ibn Sina synthesized Aristotelian logic with Islamic mathematical traditions. European algebraists refined ideas that had traveled from India through Persia. Each generation stood on foundations laid by predecessors, adding new levels of abstraction and generality.

\section*{Why This Book Exists}

As I studied, I began taking notes. These notes grew into explorations. Those explorations became chapters. Eventually, I realized I was writing a book—not the book I had planned, but the book I needed.\\
\textit{Mathesis} is my attempt to understand mathematics as computer scientists and engineers must understand it: not as pure abstraction divorced from application, nor as mere toolbox of techniques, but as living framework for systematic thought. It traces mathematical concepts from their historical origins through their modern formalizations, always asking: Why did this idea emerge? What problem did it solve? How does it connect to computation?

This book completes a trilogy of sorts:
\begin{itemize}
    \item \textit{Mathesis} provides the mathematical foundations
    \item \textit{The Art of Algorithmic Analysis} develops analytical techniques
    \item \textit{Arliz} applies these ideas to concrete data structures
\end{itemize}

Each stands alone, but together they form a coherent whole—a pathway from ancient counting to modern algorithms.

\section*{What Makes This Book Different}

Most mathematical prerequisites texts for computer science students follow a predictable pattern: rapid surveys of discrete mathematics, linear algebra, probability—topics treated as necessary evils, obstacles to overcome before "real" computer science begins. Proofs are minimized, historical context ignored, philosophical motivations unexplored.\\
This approach fails. It produces students who can manipulate mathematical symbols without understanding what those symbols mean. They can apply algorithms without grasping why those algorithms work. They memorize rather than comprehend.\\
\textit{Mathesis} takes a different path. It begins where mathematics began: with humans trying to make sense of quantity, pattern, and structure. It follows the intellectual journey from tally marks on bones to abstract algebraic structures, showing not just what we discovered but why each discovery was necessary.\\
Every major concept is developed in three ways:
\begin{itemize}
    \item \textbf{Historical}: How did this idea emerge? What problem motivated it?
    \item \textbf{Mathematical}: What is the precise, formal definition? Why this definition?
    \item \textbf{Computational}: Where does this appear in computer science? How is it used?
\end{itemize}
The goal is not merely competence but \textit{mathematical maturity}—the ability to think mathematically, to see structure where others see complexity, to recognize patterns that transcend specific contexts.

\section*{A Living Work}

Like my other books, \textit{Mathesis} is alive. It grows as my understanding deepens, as I discover new connections, as readers point out errors or suggest improvements. The version you read today will be refined next month. Next year it will be more complete.\\
This living nature is intentional. Mathematics itself is not static—new connections are constantly being discovered, old proofs simplified, different perspectives revealed. Why should a book about mathematics be frozen in time?\\
I commit to continuous improvement: clearer explanations, better examples, deeper insights. Your engagement—through corrections, suggestions, and questions—helps this process. We learn together.

\section*{How to Read This Book}

This book is designed for multiple audiences and multiple reading styles:

\textbf{For students}: Work through systematically. Do the exercises. Build understanding incrementally. This is a foundation that will serve your entire career.
\textbf{For practitioners}: Focus on parts relevant to your work. Use this as a reference when deeper mathematical understanding is needed. Return to foundations when intuition fails.
\textbf{For instructors}: Use this as a supplement to standard texts. The historical and philosophical context can motivate students in ways that purely technical presentations cannot.
\textbf{For self-learners}: Follow your curiosity. Skip sections that don't immediately interest you. Return when ready. The book will wait.\\
Most importantly: be patient with yourself. Mathematical understanding develops slowly. Confusion is not failure—it is the first stage of learning. Persist through difficulty. The rewards justify the effort.

\section*{Acknowledgment}

This book owes debts to thinkers separated by millennia: to Aristotle for showing that thought itself can be systematized; to Al-Khwarizmi for demonstrating that symbolic manipulation can solve problems; to Ibn Sina for integrating mathematics into comprehensive philosophical systems; to Descartes for making geometry algebraic; to Leibniz for dreaming of universal mathematical language; to Turing for showing that mathematics could be mechanized.\\
More immediately, I thank the readers of my other books whose questions and insights helped me understand what I had missed. Your engagement made me a better writer and thinker.

\section*{Final Thoughts}

Mathematics is hard. It should be hard—we are training our minds to think in ways that don't come naturally, to see abstractions that don't exist in physical world, to follow chains of reasoning that extend far beyond immediate intuition.\\
But mathematics is also beautiful. When you finally understand a proof, when a pattern suddenly becomes clear, when disparate concepts unite into coherent theory—those moments justify every frustration that preceded them.\\
This book is my attempt to share both the difficulty and the beauty. To show not just mathematical results but the intellectual journey that produced them. To help you develop not just mathematical knowledge but mathematical intuition.\\

Welcome to \textbf{Mathesis}. Let us begin at the beginning.

\vfill{}

\begin{flushright}
    \textit{Mahdi} \\
    \textit{2025} \\
\end{flushright}

\clearpage