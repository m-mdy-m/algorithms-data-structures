\chapter*{Preface}
\addcontentsline{toc}{chapter}{Preface}

\lettrine{I}{  kept hitting} a wall. Not in my code—that worked fine. Not in algorithms—I could trace through them step by step. The wall was deeper. It was in understanding \textit{why} things worked.

You know that feeling when you follow a proof mechanically, nodding along, getting the right answer... but something's missing? That was me. I could solve problems. I couldn't \textit{see} them.

Writing \textit{Arliz} and \textit{The Art of Algorithmic Analysis} forced me to confront this. How can you explain something you don't truly understand? I'd write a section on recurrence relations, realize I was just regurgitating formulas, delete it, and start over. Again. And again.

So I stopped. Put everything on hold. Went back to the beginning.

Not back to "Intro to Discrete Math." Further. Back to Aristotle trying to formalize human reasoning. To al-Khwarizmi figuring out how to solve equations systematically. To Euclid asking "what can we build from almost nothing?" To Ibn Sina synthesizing Greek logic with Islamic mathematics. To Descartes having his crazy idea that geometry and algebra were the same thing.

I read their actual works. Not summaries. Not textbooks about them. Their words.

And something clicked.

These people weren't just discovering math. They were \textit{thinking}—wrestling with hard problems, making wrong turns, having insights, building frameworks. Mathematics wasn't this pristine thing handed down from on high. It was messy. Human. Incomplete. Always evolving.

That's what I want to share here.

\section*{What This Book Is}

\textit{Mathesis} is my attempt to understand mathematics the way it actually developed—as a series of insights, each solving a real problem or answering a genuine question. Not "here are 50 formulas to memorize" but "here's why someone needed this idea, here's what they were trying to do, here's how it connects to everything else."

This book sits between three others I'm writing:
\begin{itemize}
    \item \textit{Mathesis} — the mathematical foundations
    \item \textit{The Art of Algorithmic Analysis} — how to analyze algorithms rigorously
    \item \textit{Arliz} — data structures in depth
\end{itemize}

You can read any of them independently. But together they form a path from "what is a number?" to "how do we build efficient software?"

\section*{How It's Different}

Most math-for-CS books treat mathematics as vegetables you have to eat before dessert. Get through the boring prerequisite chapters, then you can do the fun stuff.

That's backwards.

Mathematics \textit{is} the fun stuff. It's just taught badly.

I'm not going to give you formulas without context. Every major idea in here starts with a question: What problem were people trying to solve? Why did existing tools fail? What insight made progress possible?

Sometimes that means historical context—seeing how Babylonians tackled problems differently than Greeks, or how Islamic scholars built bridges between cultures. Sometimes it means showing failed approaches that seem reasonable but don't work. Sometimes it means proving something rigorously because the proof itself is enlightening.

The goal isn't to turn you into a mathematician. It's to give you \textit{mathematical intuition}—the ability to look at a problem and think "oh, this is really about X" or "I bet Y technique would work here."

\section*{Who Helped (Across Centuries)}

I owe debts to people I'll never meet. Aristotle for showing thought could be systematic. Al-Khwarizmi for making algebra algorithmic. Ibn Sina for treating math as part of a bigger intellectual picture. Descartes for unifying geometry and algebra. Leibniz for dreaming of universal logical languages. Turing for proving limits of computation.

Also to readers of my other books who asked questions that made me realize I didn't understand something as well as I thought. You made this better.

\section*{A Warning}

This is hard. Not "memorize 100 formulas" hard. "Change how you think" hard.

There will be moments where your brain hurts. Where a proof seems impossible to follow. Where you read the same paragraph five times and still don't get it.

That's normal. That's the process. Every mathematician goes through it.

The reward? Eventually—might take days, might take months—something clicks. Patterns emerge. Connections form. You start seeing structure everywhere. It's worth it.

\section*{One More Thing}

This book is alive. I keep learning. Readers point out mistakes. I find better ways to explain things. New connections become clear.

So the version you're reading now isn't finished. It's a snapshot of current understanding. Check back in six months and parts will be better. That's the nature of genuine learning—it never stops.

Let's begin.

\vfill
\begin{flushright}
\textit{Mahdi}\\
\textit{2025}
\end{flushright}

\clearpage