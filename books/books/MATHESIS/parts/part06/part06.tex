
\part{The Renaissance Mathematical Revolution}
\label{part:renaissance}

\begin{partintro}
\lettrine[lines=3]{T}{he Renaissance} unleashed mathematical creativity of unprecedented scope. The development of symbolic algebra transformed mathematics from geometric and rhetorical reasoning into symbolic manipulation. The invention of analytic geometry unified algebra and geometry, revealing deep connections between equations and curves. The solution of cubic and quartic equations demonstrated that systematic algebraic methods could solve problems that had resisted Greek geometry.

This part traces these revolutionary developments: Viète's symbolic algebra, Cardano's solution methods, Descartes' analytical geometry, and the broader cultural and intellectual context that made such innovations possible. We examine how new notational systems enabled new mathematical thought, and how Renaissance mathematics prepared the ground for the calculus revolution.

\vspace{1em}
\textbf{What Makes This Different:}
\begin{itemize}[noitemsep]
    \item \textbf{Symbolic Revolution:} How notation changed what could be thought
    \item \textbf{Algebraic-Geometric Unity:} The emergence of coordinate systems and analytical methods
    \item \textbf{Solution Systematization:} General methods replacing case-by-case geometric arguments
    \item \textbf{Cultural Context:} How Renaissance humanism and artisanal practice influenced mathematics
\end{itemize}

\begin{quote}
\textit{``Ars magna, the great art, is the art of solving equations of the third and fourth degree.''}

\hfill--- \textsc{Gerolamo Cardano}
\end{quote}
\end{partintro}

\chapter{The Abbacus Tradition and Practical Algebra  }
\chapter{The Cubic Equation and del Ferro-Tartaglia-Cardano  }
\chapter{Ferrari and the Solution of the Quartic  }
\chapter{Bombelli and the Acceptance of Complex Numbers  }
\chapter{François Viète and Symbolic Algebra  }
\chapter{The Development of Algebraic Notation  }
\chapter{Simon Stevin and Decimal Fractions  }
\chapter{John Napier and the Invention of Logarithms  }
\chapter{René Descartes and Analytical Geometry  }
\chapter {Pierre de Fermat and Number Theory  }
\chapter {Mathematical Perspective in Renaissance Art  }
\chapter {The Integration of Algebra and Geometry}
