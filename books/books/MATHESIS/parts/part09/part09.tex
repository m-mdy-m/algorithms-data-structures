%----------------------------------------------
\part{Formal Languages and Automata Theory}
\label{part:formal-languages}

\begin{partintro}
\lettrine[lines=3]{C}{omputation is} symbolic manipulation following formal rules. This part develops the theory of formal languages, automata, and computability—the mathematical foundations of what computers can and cannot do.

\vspace{1em}
\textbf{What Makes This Different:}
\begin{itemize}[noitemsep]
    \item \textbf{Philosophical Depth:} What is computation?
    \item \textbf{Chomsky Hierarchy:} The structure of syntactic complexity
    \item \textbf{Decidability:} The limits of algorithmic solvability
    \item \textbf{Complexity Theory:} P vs NP and beyond
\end{itemize}

\begin{quote}
\textit{``Computer science is no more about computers than astronomy is about telescopes.''}

\hfill--- \textsc{Edsger Dijkstra}
\end{quote}
\end{partintro}

\chapter{Finite Automata and Regular Languages}
\chapter{Context-Free Grammars and Pushdown Automata}
\chapter{Turing Machines and Computability}
\chapter{The Church-Turing Thesis}
\chapter{Decidability and the Halting Problem}
\chapter{Reducibility and Undecidability}
\chapter{Complexity Theory: Time and Space Classes}
\chapter{P vs NP: The Millennium Problem}
\chapter{NP-Completeness and Cook's Theorem}
\chapter{Approximation Algorithms and Hardness}
\chapter{Randomized Complexity Classes}
\chapter{Interactive Proofs and Zero-Knowledge}
