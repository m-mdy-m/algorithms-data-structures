%----------------------------------------------
\part{Topology: Continuity and Connectedness}
\label{part:topology}

\begin{partintro}
\lettrine[lines=3]{T}{opology studies} properties preserved under continuous deformation. While initially abstract, topological thinking appears throughout computer science: fixed-point theorems in semantics, computational topology in data analysis, homotopy type theory in foundations.

\vspace{1em}
\textbf{What Makes This Different:}
\begin{itemize}[noitemsep]
    \item \textbf{Computational Topology:} Persistent homology and TDA
    \item \textbf{Homotopy Type Theory:} Topological foundations for CS
    \item \textbf{Fixed-Point Theorems:} Applications to program semantics
    \item \textbf{Metric Spaces:} Foundations for analysis and optimization
\end{itemize}

\begin{quote}
\textit{``Topology is the mathematics of continuity.''}

\hfill--- \textsc{Henri Poincaré}
\end{quote}
\end{partintro}

\chapter{Metric Spaces and Topological Spaces}
\chapter{Continuity and Homeomorphisms}
\chapter{Compactness and Connectedness}
\chapter{Separation Axioms and Metrizability}
\chapter{Fundamental Group and Covering Spaces}
\chapter{Homology and Cohomology Theory}
\chapter{Manifolds and Differential Topology}
\chapter{Knot Theory and Low-Dimensional Topology}
\chapter{Fixed-Point Theorems and Applications}
\chapter{Computational Topology and Persistent Homology}
\chapter{Homotopy Type Theory}