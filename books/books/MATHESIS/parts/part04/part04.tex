%----------------------------------------------
\part{Number Theory: The Integers and Their Mysteries}
\label{part:number-theory}

\begin{partintro}
\lettrine[lines=3]{N}{umber theory}—once dismissed as the purest of pure mathematics—now underpins modern cryptography, pseudorandom generation, and algorithmic complexity. This part develops both classical and computational number theory, from Euclid's algorithm to elliptic curves.

\vspace{1em}
\textbf{What Makes This Different:}
\begin{itemize}[noitemsep]
    \item \textbf{Computational Focus:} Complexity analysis of every algorithm
    \item \textbf{Cryptographic Applications:} RSA, Diffie-Hellman, ECC in depth
    \item \textbf{Analytic Methods:} Connection to complex analysis and the Riemann hypothesis
    \item \textbf{Algorithmic Number Theory:} Primality, factorization, discrete logarithm
\end{itemize}

\begin{quote}
\textit{``Mathematics is the queen of sciences, and number theory is the queen of mathematics.''}

\hfill--- \textsc{Carl Friedrich Gauss}
\end{quote}
\end{partintro}

\chapter{Divisibility and the Fundamental Theorem of Arithmetic}
\chapter{Modular Arithmetic and Congruences}
\chapter{The Euclidean Algorithm and Its Complexity}
\chapter{Chinese Remainder Theorem}
\chapter{Fermat's Little Theorem and Euler's Theorem}
\chapter{Primality Testing: Fermat, Miller-Rabin, AKS}
\chapter{Integer Factorization: Trial Division to Number Field Sieve}
\chapter{Quadratic Residues and Legendre Symbols}
\chapter{Continued Fractions and Diophantine Approximation}
\chapter{Elliptic Curves and Cryptography}
\chapter{Analytic Number Theory: The Prime Number Theorem}
\chapter{The Riemann Hypothesis and Zeta Function}