\part{Ancient Number Systems and Positional Notation}
\label{part:ancient-systems}

\begin{partintro}
\lettrine[lines=3]{W}{ith settled civilizations} came new mathematical demands. Agricultural surplus required accounting; astronomical observation demanded precision; architecture necessitated geometric sophistication. The ancient world responded with remarkably diverse mathematical systems, each reflecting the unique needs and insights of its culture.

This part examines the major mathematical traditions of antiquity: Mesopotamian sexagesimal notation, Egyptian hieroglyphic numbers and unit fractions, the revolutionary Chinese rod calculus and matrix methods, and the sophisticated Indian numeral system that would transform world mathematics. We explore not merely their computational techniques, but the conceptual frameworks that made such techniques possible.

\vspace{1em}
\textbf{What Makes This Different:}
\begin{itemize}[noitemsep]
    \item \textbf{Comparative Analysis:} We examine why different cultures developed distinct mathematical approaches
    \item \textbf{Positional Revolution:} The conceptual leap from concrete to abstract representation
    \item \textbf{Computational Practice:} How ancient peoples actually performed calculations
    \item \textbf{Cultural Transmission:} The paths by which mathematical knowledge spread across civilizations
\end{itemize}

\begin{quote}
\textit{``I have found a very great number of exceedingly beautiful theorems.''}

\hfill--- \textsc{Archimedes, as reported by Plutarch}
\end{quote}
\end{partintro}

\chapter{Sumerian Cuneiform and Base-60 Mathematics  }
\chapter{Babylonian Mathematical Tablets and Algorithmic Procedures  }
\chapter{The Concept of Place Value and Positional Notation  }
\chapter{Egyptian Hieroglyphic Numbers and Unit Fractions  }
\chapter{The Rhind Papyrus and Systematic Problem-Solving  }
\chapter{Egyptian Geometry and Practical Mathematics  }
\chapter{Chinese Rod Numerals and Counting Boards  }
\chapter{The Nine Chapters and Matrix Operations  }
\chapter{Indus Valley Weights, Measures, and Standardization  }
\chapter{Mayan Vigesimal System and Independent Zero}
