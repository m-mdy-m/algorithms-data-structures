%----------------------------------------------
\part{Set Theory: The Language of Mathematical Objects}
\label{part:set-theory}

\begin{partintro}
\lettrine[lines=3]{S}{ets are} the atoms of mathematical discourse. Every mathematical object—numbers, functions, spaces, categories—is ultimately constructed from sets. This part develops axiomatic set theory from ZFC, explores the paradoxes that necessitate axiomatization, and examines the philosophical implications of mathematical existence.

\vspace{1em}
\textbf{What Makes This Different:}
\begin{itemize}[noitemsep]
    \item \textbf{Axiomatic Rigor:} Full development of Zermelo-Fraenkel set theory
    \item \textbf{Philosophical Context:} What does mathematical existence mean?
    \item \textbf{Computational Relevance:} Sets as data structures, type theory
    \item \textbf{Foundations of Infinity:} Cantor's paradise and its implications
\end{itemize}

\begin{quote}
\textit{``No one shall expel us from the Paradise that Cantor has created.''}

\hfill--- \textsc{David Hilbert}
\end{quote}
\end{partintro}

\chapter{Naive Set Theory and Its Paradoxes}
\chapter{Axiomatic Set Theory: ZFC}
\chapter{Relations, Functions, and Mappings}
\chapter{Cardinality and the Arithmetic of Infinity}
\chapter{Ordinal Numbers and Transfinite Induction}
\chapter{The Axiom of Choice and Its Equivalents}
\chapter{Large Cardinals and the Set-Theoretic Universe}
\chapter{Constructible Universe and Forcing}
\chapter{Set Theory and Type Theory}
\chapter{Alternative Foundations: Category Theory and HoTT}