%----------------------------------------------
\part{Discrete Mathematics: Combinatorics and Graph Theory}
\label{part:discrete}

\begin{partintro}
\lettrine[lines=3]{D}{iscrete mathematics} is the native language of computer science. Unlike continuous mathematics, we deal with countable, finite, or denumerable structures: graphs, permutations, recursive sequences. This part develops the combinatorial and graph-theoretic foundations essential for algorithm design and analysis.

\vspace{1em}
\textbf{What Makes This Different:}
\begin{itemize}[noitemsep]
    \item \textbf{Algorithmic Emphasis:} Every result connects to computation
    \item \textbf{Generating Functions:} Systematic enumeration techniques
    \item \textbf{Graph Algorithms:} From Euler to modern network science
    \item \textbf{Ramsey Theory:} The mathematics of inevitable structure
\end{itemize}

\begin{quote}
\textit{``Combinatorics is an honest subject. No adèles, no sigma-algebras. You count balls in a box, and you either have the right number or you haven't.''}

\hfill--- \textsc{Gian-Carlo Rota}
\end{quote}
\end{partintro}

\chapter{Fundamental Counting Principles}
\chapter{Permutations, Combinations, and Binomial Coefficients}
\chapter{Generating Functions and Recurrence Relations}
\chapter{The Inclusion-Exclusion Principle}
\chapter{Pigeonhole Principle and Ramsey Theory}
\chapter{Graph Theory: Foundations and Representations}
\chapter{Trees and Spanning Trees}
\chapter{Connectivity, Paths, and Cycles}
\chapter{Graph Coloring and Chromatic Numbers}
\chapter{Planar Graphs and Euler's Formula}
\chapter{Network Flows and Matching Theory}
\chapter{Spectral Graph Theory}
\chapter{Random Graphs and Probabilistic Methods}
\chapter{Extremal Combinatorics}