\part{Medieval European Mathematics}
\label{part:medieval-europe}

\begin{partintro}
\lettrine[lines=3]{M}{edieval Europe} received Greek and Islamic mathematics through translation, gradually absorbing and extending these traditions. The rise of universities, the development of systematic educational curricula, and the needs of commerce and architecture drove mathematical innovation. Though often dismissed as a period of stagnation, the medieval era laid crucial institutional and intellectual foundations for the Renaissance explosion of mathematical creativity.

This part examines how European scholars engaged with inherited mathematical traditions, how monastic and university education systematized mathematical knowledge, and how practical needs—navigation, commerce, architecture—drove theoretical advances. We explore the slow but crucial development of mathematical notation and the gradual shift toward algebraic thinking.

\vspace{1em}
\textbf{What Makes This Different:}
\begin{itemize}[noitemsep]
    \item \textbf{Institutional Context:} How universities shaped mathematical development
    \item \textbf{Translation Movement:} The transmission of Greek and Arabic texts to Latin Europe
    \item \textbf{Practical Mathematics:} Commercial arithmetic and its theoretical implications
    \item \textbf{Notational Evolution:} The gradual development of symbolic mathematical language
\end{itemize}

\begin{quote}
\textit{``In omni doctrina et scientia delectabili et utili, quam nullus ignorare debet...''}

\hfill--- \textsc{Leonardo Fibonacci, Liber Abaci}
\end{quote}
\end{partintro}

\chapter{The Translation Movement and Arabic to Latin Mathematical Transfer  }
\chapter{Monastic Mathematics and the Preservation of Knowledge  }
\chapter{The Quadrivium and Systematic Mathematical Education  }
\chapter{Fibonacci and the Introduction of Hindu-Arabic Numerals to Europe  }
\chapter{The Liber Abaci and Practical Mathematical Methods  }
\chapter{Scholastic Method and Mathematical Reasoning  }
\chapter{Nicole Oresme and Graphical Representation  }
\chapter{The Merton Calculators and Kinematics  }
\chapter{Medieval Islamic Influence on European Mathematics  }
\chapter {Commercial Mathematics and Double-Entry Bookkeeping}
