\part{Origins of Mathematical Thought}
\label{part:origins}

\begin{partintro}
\lettrine[lines=3]{M}{athematics did not} emerge fully formed from human minds. It was forged through millennia of necessity, observation, and intellectual struggle. Before symbols existed, before numbers had names, our ancestors confronted the fundamental challenge: how to comprehend and communicate quantity, pattern, and structure.

This part traces mathematics from its primordial origins—when humanity first distinguished "one" from "many"—through the revolutionary abstractions that made systematic thought possible. We examine not merely what ancient peoples calculated, but how they reasoned, what cognitive leaps enabled mathematical thinking, and why certain cultures developed particular mathematical frameworks.

\vspace{1em}
\textbf{What Makes This Different:}
\begin{itemize}[noitemsep]
    \item \textbf{Cognitive Foundations:} We explore the neurological and psychological basis for mathematical intuition
    \item \textbf{Archaeological Evidence:} Physical artifacts reveal how abstract concepts became material reality
    \item \textbf{Cultural Contexts:} Mathematical systems emerged from specific human needs and worldviews
    \item \textbf{Conceptual Evolution:} We trace how simple counting became sophisticated abstraction
\end{itemize}

\begin{quote}
\textit{``The numbers are a match for the transcendent world, and the transcendent world is a match for the numbers.''}

\hfill--- \textsc{Aristotle, Metaphysics}
\end{quote}
\end{partintro}

\chapter{Purpose and Scope of This Book}

\textbf{Why does this book exist?} Not every discipline requires a dedicated text on its analytical methods. Chemistry students learn analysis through chemical applications; physicists learn mathematical methods through physics problems. Yet algorithmic analysis merits its own focused treatment. This chapter explains why and establishes what this book does—and crucially, does not—attempt to achieve.

\section{What This Book Covers}

This book provides comprehensive coverage of techniques for analyzing algorithmic efficiency. The scope is deliberately broad, spanning from foundational concepts to advanced research-level topics.

\subsection{Asymptotic Analysis Framework}

At the heart of algorithmic analysis lies asymptotic notation—the mathematical language for describing function growth rates. We cover:

\begin{itemize}
    \item \textbf{The complete family of asymptotic notations}: Big-O ($O$), Big-Omega ($\Omega$), Big-Theta ($\Theta$), little-o ($o$), and little-omega ($\omega$)
    \item \textbf{Precise formal definitions}: Moving beyond informal intuitions to rigorous mathematical characterizations
    \item \textbf{Proof techniques}: How to establish asymptotic relationships and avoid common errors
    \item \textbf{Comparative analysis}: Understanding relative growth rates of common functions
\end{itemize}
We don't merely define notation—we develop deep understanding of \textit{why} asymptotic analysis provides the right abstraction level for comparing algorithms and \textit{when} it fails to capture important performance distinctions.

\subsection{Recurrence Analysis}

Recursive algorithms dominate computer science, making recurrence relations essential analytical tools. Our treatment includes:

\begin{itemize}
    \item \textbf{Multiple solution methods}: Substitution, recursion trees, Master Theorem, Akra-Bazzi method
    \item \textbf{Generating functions}: Powerful techniques for solving complex recurrences
    \item \textbf{Full-history recurrences}: Analyzing algorithms that depend on entire computation history
    \item \textbf{Probabilistic recurrences}: Handling randomized algorithms with recurrence-based analysis
\end{itemize}
The goal is not mere mechanical application but understanding the structure of recursive cost—why different recursion patterns produce characteristic growth rates.

\subsection{Best, Worst, and Average-Case Analysis}

Real algorithms behave differently on different inputs. We develop systematic frameworks for:

\begin{itemize}
    \item \textbf{Defining input distributions}: Formalizing what "typical" or "worst-case" means
    \item \textbf{Expected running time}: Rigorous probabilistic analysis using indicator random variables
    \item \textbf{Randomized algorithms}: Distinguishing probabilistic input analysis from algorithmic randomization
    \item \textbf{Smoothed analysis}: Modern techniques that explain why some algorithms with poor worst-case performance work well in practice
\end{itemize}

\subsection{Amortized Analysis}

Some operations are occasionally expensive but infrequent. Amortized analysis captures this by analyzing sequences of operations rather than individual operations. We cover all three major methods:

\begin{itemize}
    \item \textbf{Aggregate analysis}: Bounding total cost across operation sequences
    \item \textbf{Accounting method}: Using credit systems to track cost distribution
    \item \textbf{Potential method}: The most powerful and general amortized analysis framework
\end{itemize}
Applications include dynamic arrays, splay trees, Fibonacci heaps, and union-find structures—data structures whose efficiency depends crucially on amortized rather than worst-case analysis.

\subsection{Space Complexity}

While time complexity dominates algorithm analysis, space usage is equally important. We examine:

\begin{itemize}
    \item \textbf{Memory models}: Distinguishing auxiliary space from total space
    \item \textbf{Recursive space analysis}: Understanding call stack depth
    \item \textbf{Space-time tradeoffs}: When using more memory improves time efficiency
    \item \textbf{Streaming algorithms}: Achieving sublinear space through clever approximation
\end{itemize}

\subsection{Memory Hierarchy and I/O Complexity}

Modern performance increasingly depends on memory system behavior. We develop:

\begin{itemize}
    \item \textbf{External memory model}: Analyzing algorithms that don't fit in main memory
    \item \textbf{Cache-aware analysis}: Accounting for memory hierarchy effects
    \item \textbf{Cache-oblivious algorithms}: Techniques that perform well across all cache sizes
    \item \textbf{Parallel and multicore considerations}: How cache coherence affects algorithm design
\end{itemize}

\subsection{Lower Bounds Theory}

Understanding what's achievable requires knowing what's impossible. We cover:

\begin{itemize}
    \item \textbf{Comparison-based lower bounds}: Why sorting requires $\Omega(n \log n)$ comparisons
    \item \textbf{Adversary arguments}: Proving lower bounds through worst-case construction
    \item \textbf{Algebraic and information-theoretic bounds}: Techniques beyond comparison models
    \item \textbf{Reduction-based lower bounds}: Using problem hardness to establish limits
\end{itemize}

\subsection{Algorithm Paradigm Analysis}

Different algorithmic approaches require different analytical techniques:

\begin{itemize}
    \item \textbf{Divide-and-conquer}: Recurrence-based analysis of recursive decomposition
    \item \textbf{Dynamic programming}: State space and transition analysis
    \item \textbf{Greedy algorithms}: Correctness proofs and optimality arguments
    \item \textbf{Approximation algorithms}: Analyzing solution quality for intractable problems
\end{itemize}

\subsection{Advanced Topics}

The book extends to research-level material:

\begin{itemize}
    \item \textbf{Online algorithms}: Competitive analysis for algorithms without future knowledge
    \item \textbf{Parameterized complexity}: Fixed-parameter tractability and kernelization
    \item \textbf{Parallel algorithms}: Work-span analysis and scheduling theory
\end{itemize}

\section{What This Book Does Not Cover}

Clarity about scope requires honesty about limitations. This book deliberately excludes certain topics:

\subsection{Specific Algorithm Implementations}

This is not an algorithms encyclopedia. We use algorithms as examples to illustrate analytical techniques, but we do not attempt comprehensive coverage of all known algorithms. For extensive algorithm catalogs, consult:
\begin{itemize}
    \item Cormen et al., \textit{Introduction to Algorithms}
    \item Sedgewick and Wayne, \textit{Algorithms}
    \item Skiena, \textit{The Algorithm Design Manual}
\end{itemize}
Our focus remains on \textit{how to analyze} algorithms, not cataloging \textit{which} algorithms exist.

\subsection{Programming Language Specifics}

Pseudocode appears throughout, but we avoid language-specific implementations. Analysis techniques apply regardless of implementation language. When performance depends on language features (garbage collection, memory management), we discuss the abstract impact but not language-specific details.

\subsection{Empirical Performance Engineering}

We bridge theory and practice, but detailed empirical optimization (profiling, compiler optimization, architecture-specific tuning) exceeds our scope. These topics merit their own books. We focus on analytical prediction rather than empirical measurement.

\subsection{Complete Complexity Theory}

While we introduce computational complexity concepts (P, NP, NP-completeness), this book is not a complexity theory text. For comprehensive treatment, see:
\begin{itemize}
    \item Sipser, \textit{Introduction to the Theory of Computation}
    \item Arora and Barak, \textit{Computational Complexity: A Modern Approach}
\end{itemize}
We cover complexity theory sufficient for algorithm analysis but not as a primary focus.

\subsection{Advanced Probability Theory}

Our probabilistic analysis uses elementary probability—random variables, expectations, basic inequalities. We don't require measure theory, martingales, or advanced stochastic processes. For algorithms requiring deeper probability theory, we provide references but don't develop the theory ourselves.

\subsection{Numerical and Scientific Computing}

Numerical stability, floating-point arithmetic, and scientific computing algorithms have specialized analysis techniques. While we touch on these in examples, dedicated numerical analysis texts provide comprehensive treatment.

\subsection{Cryptographic and Security Considerations}

Security analysis requires different frameworks—computational hardness assumptions, adversary models, provable security reductions. These warrant separate study. We analyze cryptographic algorithms' efficiency but not their security properties.

\section{Target Audience: Students, Researchers, and Practitioners}

This book serves multiple communities with overlapping but distinct needs.

\subsection{Undergraduate Computer Science Students}

\paragraph{Background Assumed:}
You've completed introductory programming (CS1/CS2), basic data structures (CS2/CS3), and ideally an algorithms course. You're comfortable with:
\begin{itemize}
    \item Programming in at least one language (Java, Python, C++, etc.)
    \item Basic data structures (arrays, linked lists, trees, hash tables)
    \item Elementary discrete mathematics (sets, functions, basic counting)
    \item Introductory proof techniques (induction, contradiction)
\end{itemize}

\paragraph{What You'll Gain:}
\begin{itemize}
    \item Rigorous foundation for understanding algorithmic efficiency
    \item Mathematical tools for comparing algorithm performance
    \item Preparation for advanced algorithms courses
    \item Framework for analyzing data structures and algorithms in future coursework
    \item Skills for technical interviews that probe algorithmic thinking
\end{itemize}

\paragraph{How to Use This Book:}
Work through systematically, focusing especially on Chapters 1-4 (asymptotic analysis, recurrences, best/worst/average case). Complete exercises—they're essential for developing analytical skills. Parts III-V provide enrichment but aren't required for foundational understanding.

\subsection{Graduate Students in Computer Science}

\paragraph{Background Assumed:}
Solid undergraduate algorithms education. Comfort with mathematical proofs, probability theory, and abstract thinking. Experience implementing data structures and algorithms.

\paragraph{What You'll Gain:}
\begin{itemize}
    \item Advanced analytical techniques for research-level work
    \item Preparation for reading algorithms research papers
    \item Frameworks for analyzing novel algorithms in your research
    \item Understanding of analytical methods' strengths and limitations
    \item Bridge between undergraduate algorithms and theoretical CS research
\end{itemize}

\paragraph{How to Use This Book:}
You may skim early chapters if you're confident in fundamentals, but don't skip review material entirely—even experienced students find perspective-shifting insights. Focus on Parts III-V and advanced topics. Engage deeply with exercises, especially proof-based problems. Use the book as reference when analyzing algorithms in your research.

\subsection{Practitioners and Software Engineers}

\paragraph{Background Assumed:}
Professional programming experience. Practical familiarity with data structures and algorithms, even if formal training was limited. Comfort with quantitative reasoning and learning from technical material.

\paragraph{What You'll Gain:}
\begin{itemize}
    \item Rigorous framework for algorithm selection decisions
    \item Understanding of why certain algorithms are "efficient"
    \item Tools for predicting performance at scale
    \item Ability to analyze custom algorithms and data structures
    \item Vocabulary for discussing algorithm efficiency with colleagues
    \item Foundation for understanding algorithm optimization literature
\end{itemize}

\paragraph{How to Use This Book:}
Focus on Parts I-II initially, emphasizing intuition over formal proofs. Work through examples carefully—they connect theory to practice. Later parts provide depth when needed for specific problems. Use as reference when choosing between algorithmic approaches or diagnosing performance issues.

\subsection{Self-Learners and Independent Scholars}

\paragraph{Background Assumed:}
Strong intellectual curiosity. Comfort with mathematical thinking and learning independently. Programming experience helpful but not strictly required for analytical techniques.

\paragraph{What You'll Gain:}
\begin{itemize}
    \item Systematic understanding of how computer scientists reason about efficiency
    \item Mathematical literacy in algorithmic analysis
    \item Ability to read and understand algorithms research
    \item Framework for evaluating algorithm descriptions in technical literature
    \item Intellectual satisfaction of understanding deep theoretical foundations
\end{itemize}

\paragraph{How to Use This Book:}
Proceed at your own pace. Don't rush—genuine understanding takes time. Engage actively with exercises even without formal accountability. Join online communities (see Appendix F) for discussion and clarification. Consider the book a long-term companion rather than a quick read.

\subsection{Researchers in Adjacent Fields}

\paragraph{Background Assumed:}
Strong quantitative background in a related field (mathematics, physics, operations research, bioinformatics). Need for algorithmic analysis tools to support research in your primary area.

\paragraph{What You'll Gain:}
\begin{itemize}
    \item Computer science perspective on computational efficiency
    \item Tools for analyzing algorithms in your research domain
    \item Understanding of when and why algorithmic costs matter
    \item Bridge between your field's analytical methods and CS techniques
\end{itemize}

\paragraph{How to Use This Book:}
Focus on concepts most relevant to your work. The modular structure allows selective reading. Mathematical background may let you move quickly through formal material. Pay attention to connections between CS analysis and techniques in your field—cross-pollination often yields insights.

\section{Prerequisites and Preparation}

Success with this book requires certain foundations. This section helps you assess readiness and identify gaps to address.

\subsection{Essential Prerequisites}

\paragraph{Mathematical Maturity}
You should be comfortable with:
\begin{itemize}
    \item Mathematical notation and formal definitions
    \item Logical reasoning and proof structures
    \item Working with abstractions and generalizations
    \item Translating intuitive ideas into precise statements
\end{itemize}

\textit{Assessment:} If you can follow a proof by induction and understand why it works, you likely have sufficient mathematical maturity.

\paragraph{Discrete Mathematics}
Required background includes:
\begin{itemize}
    \item Sets, functions, and relations
    \item Basic graph theory (graphs, trees, paths)
    \item Elementary combinatorics (permutations, combinations, binomial coefficients)
    \item Summation notation and common summations
    \item Floor and ceiling functions, logarithms
\end{itemize}

\textit{Remediation:} Chapter 3 provides review. For deeper preparation, consult Rosen's \textit{Discrete Mathematics and Its Applications} or Lehman et al.'s \textit{Mathematics for Computer Science}.

\paragraph{Proof Techniques}
You should recognize and construct:
\begin{itemize}
    \item Direct proofs
    \item Proof by contradiction
    \item Proof by induction (weak and strong)
    \item Proof by contrapositive
\end{itemize}

\textit{Remediation:} Chapter 3, Section 1 reviews proof methods. Velleman's \textit{How to Prove It} provides excellent introduction.

\paragraph{Probability Theory}
Basic understanding of:
\begin{itemize}
    \item Sample spaces and events
    \item Probability distributions
    \item Random variables and expectations
    \item Independence and conditional probability
\end{itemize}

\textit{Remediation:} Chapter 3, Section 2 reviews probability essentials. Ross's \textit{A First Course in Probability} offers comprehensive introduction.

\subsection{Recommended but Not Essential}

\paragraph{Calculus}
Helpful for:
\begin{itemize}
    \item Understanding limits and asymptotic behavior
    \item Working with continuous approximations
    \item Some advanced analysis techniques (generating functions)
\end{itemize}

Single-variable calculus suffices; multivariable calculus rarely appears.

\paragraph{Linear Algebra}
Occasionally useful for:
\begin{itemize}
    \item Matrix operations complexity
    \item Markov chain analysis
    \item Some graph algorithms
\end{itemize}

Chapter 3, Section 4 provides sufficient review.

\paragraph{Programming Experience}
Helpful for:
\begin{itemize}
    \item Intuition about algorithm behavior
    \item Understanding implementation tradeoffs
    \item Connecting analysis to practice
\end{itemize}

Not strictly required for learning analytical techniques, but practical experience enriches understanding.

\subsection{Readiness Self-Assessment}

Before beginning, attempt these questions:

\begin{enumerate}
    \item What is the relationship between the functions $n^2$ and $2^n$ as $n$ grows large?
    \item Express using summation notation: $1 + 2 + 4 + 8 + \cdots + 2^n$
    \item If $f(n) = 3n^2 + 5n + 7$, what is the dominant term as $n \to \infty$?
    \item Prove by induction: $\sum_{i=1}^{n} i = \frac{n(n+1)}{2}$
    \item If you flip a fair coin $n$ times, what is the expected number of heads?
\end{enumerate}
If you answered most correctly, you're well-prepared. If you struggled, review prerequisite material before continuing.

\section{How to Succeed with This Book}

Learning rigorous analytical techniques requires specific strategies. This section offers guidance based on common pitfalls and successful approaches.

\subsection{Active Engagement}

\paragraph{Don't Just Read—Work}
Algorithmic analysis is not a spectator sport. Reading proofs passively provides false confidence. Instead:
\begin{itemize}
    \item Work through mathematical derivations yourself
    \item Attempt examples before reading solutions
    \item Pause at claims to verify you understand why they're true
    \item Cover solutions and try reconstructing arguments independently
\end{itemize}

\paragraph{Embrace Difficulty}
If concepts feel challenging, you're learning correctly. Comfort often signals superficial understanding. When stuck:
\begin{itemize}
    \item Persist with the difficulty rather than immediately seeking help
    \item Try explaining the concept to yourself in your own words
    \item Construct your own examples
    \item Return to earlier material to strengthen foundations
\end{itemize}

\paragraph{Make Connections}
Isolated knowledge fragments quickly fade. Constantly ask:
\begin{itemize}
    \item How does this relate to earlier concepts?
    \item Why is this technique useful?
    \item When would I choose this method over alternatives?
    \item What are the key insights, stripped of technical details?
\end{itemize}

\subsection{Exercise Strategy}

\paragraph{Attempt Every Exercise}
Exercises aren't optional review—they're integral to learning. Many exercises:
\begin{itemize}
    \item Introduce concepts later chapters assume
    \item Build problem-solving skills proofs require
    \item Reveal connections not explicit in main text
    \item Develop the analytical intuition that separates understanding from memorization
\end{itemize}

\paragraph{Struggle Before Seeking Solutions}
Solutions appear in Appendix C, but premature consultation undermines learning. Develop the habit:
\begin{itemize}
    \item Spend substantial time (hours, if needed) on challenging problems
    \item Try multiple approaches when stuck
    \item Consult earlier chapters for relevant techniques
    \item Only after genuine effort, check solutions—but then understand them deeply
\end{itemize}

\paragraph{Write Formal Solutions}
Don't settle for understanding ideas vaguely. Write complete, formal solutions:
\begin{itemize}
    \item State what you're proving clearly
    \item Justify each step explicitly
    \item Use precise mathematical language
    \item Conclude by confirming you've answered the question
\end{itemize}

This discipline builds the rigor professional work requires.

\subsection{Pacing and Persistence}

\paragraph{Don't Rush}
Deep understanding requires time. Resist pressure to:
\begin{itemize}
    \item Skip challenging sections
    \item Skim proofs without understanding
    \item Move forward with shaky foundations
    \item Prioritize coverage over comprehension
\end{itemize}

Better to thoroughly understand half the book than superficially "complete" all of it.

\paragraph{Expect Non-Linearity}
Learning advanced material isn't smoothly progressive:
\begin{itemize}
    \item Some concepts require multiple exposures before clicking
    \item Understanding often arrives suddenly after prolonged confusion
    \item Later material sometimes clarifies earlier confusion
    \item Apparent mastery may prove illusory when tested
\end{itemize}

This is normal. Persist through frustration.

\paragraph{Take Breaks Strategically}
When truly stuck:
\begin{itemize}
    \item Step away and return later—fresh perspective helps
    \item Work on different material and return with broader context
    \item Sleep on problems—subconscious processing is real
    \item But don't use breaks to avoid difficult material permanently
\end{itemize}

\subsection{Resource Utilization}

\paragraph{Use External References Judiciously}
This book is comprehensive but not encyclopedic. When seeking additional perspective:
\begin{itemize}
    \item Use references to clarify confusion, not replace effort
    \item Compare multiple sources to build robust understanding
    \item Return to this book's treatment after external exploration
    \item See Appendix F for recommended supplementary resources
\end{itemize}

\paragraph{Engage with Community}
Learning improves through discussion:
\begin{itemize}
    \item Join online forums focused on algorithms and analysis
    \item Explain concepts to others—teaching reveals understanding gaps
    \item Don't hesitate to ask questions, but show your work first
    \item Contribute corrections and improvements through GitHub
\end{itemize}

\paragraph{Maintain a Working Document}
Create personal notes:
\begin{itemize}
    \item Summarize key concepts in your own words
    \item Collect solved exercises for later review
    \item Note connections and insights as they occur
    \item Build your own example repository
\end{itemize}

This reference becomes invaluable for review and future work.

\section{A Note on Rigor}

This book takes rigor seriously. Not as pedantry, but as precision—the discipline that lets us reason correctly about complex systems.

\subsection{Why Rigor Matters}

Informal intuition is valuable but insufficient. Rigorous analysis provides:

\paragraph{Reliability}
Intuition misleads. Logarithms "feel" similar to constants. Quadratic and cubic growth "seem" comparable. Amortized and average case sound equivalent. Rigorous analysis distinguishes what intuition conflates.

\paragraph{Generality}
Precise reasoning extends beyond specific cases. A rigorous proof about comparison-based sorting applies to all such algorithms, not just examples you've seen.

\paragraph{Communication}
Mathematics provides unambiguous language. ``Fast'' is vague; $O(n \log n)$ is precise. Proofs make claims verifiable and falsifiable.
\chapter{Mathematical Foundations II: Analysis and Probability}
\label{ch:analysis-probability}

\lettrine[lines=3]{A}{lgorithm analysis} is fundamentally about asymptotic 
behavior: what happens as input size grows without bound? This requires tools 
from mathematical analysis—limits, series, asymptotic expansions. Meanwhile, 
randomized algorithms demand probability theory.

\begin{chapterintro}
\textbf{Chapter Organization:}

This chapter covers four major areas:
\begin{enumerate}[noitemsep]
    \item \textbf{Mathematical Analysis} — Limits, series, asymptotics
    \item \textbf{Probability Theory} — Measure-theoretic foundations to applications
    \item \textbf{Information Theory} — Entropy, coding, lower bounds
    \item \textbf{Number Theory} — Modular arithmetic, primality, factorization
\end{enumerate}

\textbf{Prerequisites:} Single-variable calculus (derivatives, integrals, series). 
Basic probability at the level of a first course.
\end{chapterintro}

% %==============================================
% % SECTION 1: MATHEMATICAL ANALYSIS
% %==============================================
% \section{Mathematical Analysis for Asymptotic Reasoning}
% \label{sec:mathematical-analysis}

% \begin{sectionintro}
% Algorithm analysis lives in the asymptotic realm: we care about behavior as 
% $n \to \infty$. This section develops the analytical tools—limits, series, 
% approximations—that make precise asymptotic reasoning possible.

% \textbf{Core Philosophy:} We don't just state formulas—we prove them, understand 
% their limitations, and learn when they apply.
% \end{sectionintro}

% %----------------------------------------------
% \subsection{Sequences, Series, and Limits}
% \label{subsec:sequences-series}

% \subsubsection{Sequences and Convergence}
% \paragraph{Definition of Limit: $\lim_{n \to \infty} a_n = L$}
% \begin{itemize}[noitemsep]
%     \item Formal $\varepsilon$-$N$ definition
%     \item Intuitive meaning and visualization
%     \item Example: $\lim_{n \to \infty} \frac{1}{n} = 0$
% \end{itemize}

% \paragraph{Limit Laws}
% \begin{itemize}[noitemsep]
%     \item Sum, difference, product, quotient rules
%     \item Squeeze theorem
%     \item Monotone convergence theorem
% \end{itemize}

% \paragraph{Important Limits}
% \begin{itemize}[noitemsep]
%     \item $\lim_{n \to \infty} \left(1 + \frac{1}{n}\right)^n = e$
%     \item $\lim_{n \to \infty} \frac{n^k}{a^n} = 0$ for $a > 1$
%     \item $\lim_{n \to \infty} \frac{\log n}{n^\varepsilon} = 0$ for any $\varepsilon > 0$
% \end{itemize}

% \subsubsection{Infinite Series}
% \paragraph{Definition and Convergence}
% \begin{itemize}[noitemsep]
%     \item Partial sums: $S_n = \sum_{k=1}^{n} a_k$
%     \item Convergence: $\lim_{n \to \infty} S_n$ exists
%     \item Divergence and oscillation
% \end{itemize}

% \paragraph{Convergence Tests}
% \begin{itemize}[noitemsep]
%     \item Comparison test
%     \item Ratio test (d'Alembert)
%     \item Root test (Cauchy)
%     \item Integral test
%     \item Alternating series test (Leibniz)
% \end{itemize}

% \paragraph{Absolute vs. Conditional Convergence}

% \subsubsection{Special Series}
% \paragraph{Geometric Series}
% \begin{itemize}[noitemsep]
%     \item $\sum_{k=0}^{\infty} r^k = \frac{1}{1-r}$ for $|r| < 1$
%     \item Finite sum: $\sum_{k=0}^{n} r^k = \frac{1-r^{n+1}}{1-r}$
%     \item Applications to algorithm analysis
% \end{itemize}

% \paragraph{Harmonic Series}
% \begin{itemize}[noitemsep]
%     \item $H_n = \sum_{k=1}^{n} \frac{1}{k} = \ln n + \gamma + O(1/n)$
%     \item Euler-Mascheroni constant: $\gamma \approx 0.5772$
%     \item Generalized harmonic numbers: $H_n^{(r)} = \sum_{k=1}^{n} \frac{1}{k^r}$
% \end{itemize}

% \paragraph{Power Series}
% \begin{itemize}[noitemsep]
%     \item Definition: $\sum_{n=0}^{\infty} a_n x^n$
%     \item Radius of convergence
%     \item Operations on power series
% \end{itemize}

% %----------------------------------------------
% \subsection{Asymptotic Notation (Informal Introduction)}
% \label{subsec:asymptotic-informal}

% \begin{subsectionintro}
% We introduce asymptotic notation informally here; rigorous definitions appear 
% in Part 2. This preview helps us use the notation in subsequent proofs.
% \end{subsectionintro}

% \subsubsection{Big-O Notation: $O(g(n))$}
% \paragraph{Intuition: Upper Bound}
% \begin{itemize}[noitemsep]
%     \item $f(n) = O(g(n))$ means $f$ grows no faster than $g$
%     \item Formal (preview): $\exists c, n_0 : f(n) \leq c \cdot g(n)$ for $n \geq n_0$
% \end{itemize}

% \paragraph{Common Examples}
% \begin{itemize}[noitemsep]
%     \item $3n^2 + 5n + 7 = O(n^2)$
%     \item $\log n = O(n^\varepsilon)$ for any $\varepsilon > 0$
% \end{itemize}

% \subsubsection{Big-Omega: $\Omega(g(n))$}
% \paragraph{Intuition: Lower Bound}

% \subsubsection{Big-Theta: $\Theta(g(n))$}
% \paragraph{Intuition: Tight Bound}

% \subsubsection{Little-o and Little-omega}
% \paragraph{Strict Inequalities}

% \subsubsection{Growth Rate Hierarchy}
% $$1 \prec \log \log n \prec \log n \prec \sqrt{n} \prec n \prec n \log n \prec n^2 \prec 2^n \prec n!$$

% %----------------------------------------------
% \subsection{Summation Techniques}
% \label{subsec:summations}

% \begin{subsectionintro}
% Converting sums to closed forms is a fundamental skill in complexity analysis. 
% We develop multiple techniques, from elementary to advanced.
% \end{subsectionintro}

% \subsubsection{Basic Formulas}
% \paragraph{Arithmetic Series}
% $$\sum_{k=1}^{n} k = \frac{n(n+1)}{2}$$

% \paragraph{Sum of Squares}
% $$\sum_{k=1}^{n} k^2 = \frac{n(n+1)(2n+1)}{6}$$

% \paragraph{Sum of Cubes}
% $$\sum_{k=1}^{n} k^3 = \left(\frac{n(n+1)}{2}\right)^2$$

% \paragraph{Geometric Sum}
% $$\sum_{k=0}^{n} r^k = \frac{r^{n+1} - 1}{r - 1}$$

% \subsubsection{Perturbation Method}
% \paragraph{Technique}
% \begin{enumerate}[noitemsep]
%     \item Write $S_n = \sum_{k=1}^{n} f(k)$
%     \item Consider $S_n - S_{n-1}$ or $S_n - r \cdot S_{n-1}$
%     \item Solve the resulting equation
% \end{enumerate}

% \paragraph{Example: Geometric Series Derivation}

% \subsubsection{Repertoire Method}
% \paragraph{Technique (Concrete Mathematics)}
% \begin{enumerate}[noitemsep]
%     \item Assume solution form: $S_n = A \cdot \alpha(n) + B \cdot \beta(n) + \cdots$
%     \item Test with known special cases
%     \item Solve for coefficients $A, B, \ldots$
% \end{enumerate}

% \paragraph{Example: Solving $S_n = S_{n-1} + n$}

% \subsubsection{Integral Approximation}
% \paragraph{Integral Test}
% $$\int_1^n f(x) \, dx \leq \sum_{k=1}^{n} f(k) \leq f(1) + \int_1^n f(x) \, dx$$

% \paragraph{Example: Harmonic Series}
% $$\ln n < H_n < 1 + \ln n$$

% \subsubsection{Euler-Maclaurin Formula}
% \paragraph{Statement}
% $$\sum_{k=a}^{b} f(k) = \int_a^b f(x) \, dx + \frac{f(a) + f(b)}{2} + \sum_{k=1}^{p} \frac{B_{2k}}{(2k)!} \left(f^{(2k-1)}(b) - f^{(2k-1)}(a)\right) + R_p$$

% where $B_{2k}$ are Bernoulli numbers.

% \paragraph{Applications}
% \begin{itemize}[noitemsep]
%     \item Precise asymptotics of $H_n$
%     \item Stirling's approximation derivation
%     \item Correcting integral approximations
% \end{itemize}

% \paragraph{Bernoulli Numbers}
% \begin{itemize}[noitemsep]
%     \item Generating function: $\frac{x}{e^x - 1} = \sum_{n=0}^{\infty} B_n \frac{x^n}{n!}$
%     \item First few: $B_0 = 1, B_1 = -1/2, B_2 = 1/6, B_4 = -1/30$
% \end{itemize}

% \subsubsection{Generating Function Method}
% \paragraph{See Section~\ref{subsec:generating-functions} in Chapter 1}

% %----------------------------------------------
% \subsection{Asymptotic Analysis of Functions}
% \label{subsec:asymptotic-functions}

% \subsubsection{L'Hôpital's Rule}
% \paragraph{Statement}
% If $\lim_{x \to a} f(x) = \lim_{x \to a} g(x) = 0$ or $\pm\infty$, then:
% $$\lim_{x \to a} \frac{f(x)}{g(x)} = \lim_{x \to a} \frac{f'(x)}{g'(x)}$$
% (if the right limit exists).

% \paragraph{Applications}
% \begin{itemize}[noitemsep]
%     \item Comparing growth rates
%     \item Resolving indeterminate forms: $0/0, \infty/\infty, 0 \cdot \infty$
% \end{itemize}

% \paragraph{Example: $\lim_{n \to \infty} \frac{\log n}{n}$}

% \subsubsection{Taylor Series and Expansions}
% \paragraph{Taylor Series}
% $$f(x) = \sum_{n=0}^{\infty} \frac{f^{(n)}(a)}{n!} (x-a)^n$$

% \paragraph{Maclaurin Series (at $a=0$)}

% \paragraph{Common Series}
% \begin{align*}
% e^x &= \sum_{n=0}^{\infty} \frac{x^n}{n!} \\
% \sin x &= \sum_{n=0}^{\infty} \frac{(-1)^n x^{2n+1}}{(2n+1)!} \\
% \cos x &= \sum_{n=0}^{\infty} \frac{(-1)^n x^{2n}}{(2n)!} \\
% \ln(1+x) &= \sum_{n=1}^{\infty} \frac{(-1)^{n-1} x^n}{n} \quad (|x| < 1) \\
% (1+x)^\alpha &= \sum_{n=0}^{\infty} \binom{\alpha}{n} x^n \quad (|x| < 1)
% \end{align*}

% \paragraph{Applications to Approximation}
% \begin{itemize}[noitemsep]
%     \item $e^x \approx 1 + x + \frac{x^2}{2}$ for small $x$
%     \item $\ln(1+x) \approx x - \frac{x^2}{2}$ for small $x$
% \end{itemize}

% \subsubsection{Asymptotic Expansions}
% \paragraph{Definition}
% $f(x) \sim \sum_{n=0}^{\infty} \frac{a_n}{x^n}$ as $x \to \infty$ means:
% $$f(x) = \sum_{n=0}^{N-1} \frac{a_n}{x^n} + O\left(\frac{1}{x^N}\right)$$

% \paragraph{Stirling's Approximation (Full Expansion)}
% $$n! = \sqrt{2\pi n} \left(\frac{n}{e}\right)^n \left(1 + \frac{1}{12n} + \frac{1}{288n^2} - \frac{139}{51840n^3} + O\left(\frac{1}{n^4}\right)\right)$$

% \paragraph{Ramanujan's Refinement}
% $$n! \approx \sqrt{\pi} \left(\frac{n}{e}\right)^n \sqrt{8n^3 + 4n^2 + n + \frac{1}{30}}$$

% %----------------------------------------------
% \subsection{Stirling's Approximation}
% \label{subsec:stirling-detailed}

% \begin{subsectionintro}
% Stirling's approximation is one of the most important results in asymptotic 
% analysis. We present three derivations and discuss applications.
% \end{subsectionintro}

% \subsubsection{Statement of Stirling's Formula}
% \paragraph{Simple Form}
% $$n! \sim \sqrt{2\pi n} \left(\frac{n}{e}\right)^n$$

% \paragraph{Relative Error}
% The error is: $\frac{1}{12n} + O(1/n^2)$, so extremely accurate even for small $n$.

% \subsubsection{First Derivation: Euler-Maclaurin Formula}
% \paragraph{Step 1: Logarithm of Factorial}
% $$\ln(n!) = \sum_{k=1}^{n} \ln k$$

% \paragraph{Step 2: Apply Euler-Maclaurin}
% $$\sum_{k=1}^{n} \ln k = \int_1^n \ln x \, dx + \frac{\ln 1 + \ln n}{2} + O(1)$$

% \paragraph{Step 3: Evaluate Integral}
% $$\int_1^n \ln x \, dx = n \ln n - n + 1$$

% \paragraph{Step 4: Combine and Exponentiate}
% $$\ln(n!) = n \ln n - n + \frac{1}{2} \ln n + C + O(1/n)$$

% Determining $C = \frac{1}{2} \ln(2\pi)$ requires Wallis's product.

% \subsubsection{Second Derivation: Wallis's Product}
% \paragraph{Wallis's Product Formula}
% $$\prod_{n=1}^{\infty} \frac{4n^2}{4n^2 - 1} = \frac{\pi}{2}$$

% \paragraph{Connection to Stirling}
% Use binomial coefficients $\binom{2n}{n}$ and asymptotic analysis.

% \subsubsection{Third Derivation: Laplace's Method}
% \paragraph{Idea: Approximate $n!$ via Integral}
% $$n! = \int_0^{\infty} x^n e^{-x} \, dx$$

% \paragraph{Laplace's Method}
% Approximate integral by Gaussian near maximum.

% \subsubsection{Applications}
% \paragraph{Asymptotics of Binomial Coefficients}
% $$\binom{2n}{n} \sim \frac{4^n}{\sqrt{\pi n}}$$

% \paragraph{Entropy and Information Theory}
% $$\log_2(n!) \approx n \log_2 n - n \log_2 e + \frac{1}{2} \log_2(2\pi n)$$

% \paragraph{Catalan Numbers}
% $$C_n = \frac{1}{n+1} \binom{2n}{n} \sim \frac{4^n}{n^{3/2} \sqrt{\pi}}$$

% %----------------------------------------------
% \subsection{Integration Techniques}
% \label{subsec:integration}

% \subsubsection{Fundamental Theorem of Calculus}
% \subsubsection{Integration by Parts}
% \subsubsection{Integration by Substitution}
% \subsubsection{Partial Fractions}
% \subsubsection{Numerical Integration and Error Bounds}

% %----------------------------------------------
% \subsection{Advanced Asymptotic Methods}
% \label{subsec:advanced-asymptotics}

% \subsubsection{Saddle-Point Method}
% \paragraph{For Integral Asymptotics}

% \subsubsection{Method of Steepest Descent}

% \subsubsection{Singularity Analysis of Generating Functions}
% \paragraph{Transfer Theorems}

% \subsection{Exercises}
% \subsubsection{Warmup Problems (25)}
% \subsubsection{Standard Problems (35)}
% \subsubsection{Challenging Problems (30)}
% \subsubsection{Research Problems ($\star$) (10)}

% %==============================================
% % SECTION 2: PROBABILITY THEORY
% %==============================================
% \section{Probability Theory for Algorithmic Analysis}
% \label{sec:probability-theory}

% \begin{sectionintro}
% Randomized algorithms and average-case analysis demand rigorous probability 
% theory. We develop probability from measure-theoretic foundations to practical 
% applications, with emphasis on techniques used in algorithm analysis.

% \textbf{Approach:} We start with formal measure theory (briefly), then focus on 
% discrete probability with full proofs of major theorems.
% \end{sectionintro}

% %----------------------------------------------
% \subsection{Probability Spaces and Measure Theory}
% \label{subsec:measure-theory}

% \subsubsection{Measure-Theoretic Foundations (Overview)}
% \paragraph{$\sigma$-Algebras}
% \paragraph{Probability Measures}
% \paragraph{Random Variables as Measurable Functions}

% \subsubsection{Discrete Probability Spaces}
% \paragraph{Sample Space: $\Omega$}
% \paragraph{Events: Subsets of $\Omega$}
% \paragraph{Probability Function: $\Prob: 2^\Omega \to [0,1]$}

% \subsubsection{Kolmogorov's Axioms}
% \begin{enumerate}[noitemsep]
%     \item $\Prob(A) \geq 0$ for all events $A$
%     \item $\Prob(\Omega) = 1$
%     \item For disjoint events: $\Prob(A_1 \cup A_2 \cup \cdots) = \Prob(A_1) + \Prob(A_2) + \cdots$
% \end{enumerate}

% \subsubsection{Consequences of Axioms}
% \paragraph{$\Prob(\emptyset) = 0$}
% \paragraph{$\Prob(A^c) = 1 - \Prob(A)$}
% \paragraph{Inclusion-Exclusion for Probability}

% %----------------------------------------------
% \subsection{Conditional Probability and Independence}
% \label{subsec:conditional-probability}

% \subsubsection{Conditional Probability}
% \paragraph{Definition}
% $$\Prob(A \mid B) = \frac{\Prob(A \cap B)}{\Prob(B)}$$

% \paragraph{Intuition: Updating Beliefs}

% \subsubsection{Law of Total Probability}
% \paragraph{Partition Formula}
% If $B_1, \ldots, B_n$ partition $\Omega$:
% $$\Prob(A) = \sum_{i=1}^{n} \Prob(A \mid B_i) \Prob(B_i)$$

% \subsubsection{Bayes' Theorem}
% \paragraph{Statement}
% $$\Prob(B \mid A) = \frac{\Prob(A \mid B) \Prob(B)}{\Prob(A)}$$

% \paragraph{Bayesian Inference}
% \paragraph{Applications to Machine Learning}

% \subsubsection{Independence}
% \paragraph{Definition}
% Events $A$ and $B$ are independent if:
% $$\Prob(A \cap B) = \Prob(A) \cdot \Prob(B)$$

% \paragraph{Mutual Independence}
% \paragraph{Pairwise vs. Mutual Independence}

% %----------------------------------------------
% \subsection{Random Variables}
% \label{subsec:random-variables-detailed}

% \subsubsection{Definition}
% \paragraph{Formal: Measurable Function $X: \Omega \to \mathbb{R}$}
% \paragraph{Intuitive: Numerical Outcome of Experiment}

% \subsubsection{Probability Mass Function (PMF)}
% \paragraph{Definition: $p_X(x) = \Prob(X = x)$}
% \paragraph{Properties: $\sum_x p_X(x) = 1$}

% \subsubsection{Cumulative Distribution Function (CDF)}
% \paragraph{Definition: $F_X(x) = \Prob(X \leq x)$}
% \paragraph{Properties: Non-decreasing, right-continuous, limits}

% \subsubsection{Expected Value}
% \paragraph{Definition}
% $\Expect[X] \geq \Expect[a \cdot I] = a \cdot \Expect[I] = a \cdot \Prob(X \geq a)$
% \end{proof}

% \paragraph{Applications}
% \begin{itemize}[noitemsep]
%     \item Tail bounds for non-negative random variables
%     \item Analysis of randomized algorithms
% \end{itemize}

% \subsubsection{Chebyshev's Inequality}
% \paragraph{Statement}
% For any random variable $X$ with finite variance and $k > 0$:
% $\Prob(|X - \Expect[X]| \geq k) \leq \frac{\Var(X)}{k^2}$

% \paragraph{Alternative Form}
% $\Prob(|X - \Expect[X]| \geq k\sigma) \leq \frac{1}{k^2}$

% \paragraph{Proof}
% \begin{proof}
% Apply Markov's inequality to $(X - \Expect[X])^2$:
% $\Prob(|X - \Expect[X]| \geq k) = \Prob((X - \Expect[X])^2 \geq k^2) \leq \frac{\Expect[(X - \Expect[X])^2]}{k^2} = \frac{\Var(X)}{k^2}$
% \end{proof}

% \paragraph{Example: Balls into Bins}
% With $n$ balls thrown into $n$ bins uniformly, the maximum load is 
% $O\left(\frac{\log n}{\log \log n}\right)$ with high probability.

% \subsubsection{Chernoff Bounds}
% \paragraph{Statement (Multiplicative Form)}
% Let $X = \sum_{i=1}^{n} X_i$ where $X_i$ are independent Bernoulli r.v.s. 
% Let $\mu = \Expect[X]$. Then:
% \begin{itemize}[noitemsep]
%     \item Upper tail: $\Prob(X \geq (1+\delta)\mu) \leq e^{-\frac{\delta^2 \mu}{3}}$ for $0 < \delta < 1$
%     \item Lower tail: $\Prob(X \leq (1-\delta)\mu) \leq e^{-\frac{\delta^2 \mu}{2}}$ for $0 < \delta < 1$
% \end{itemize}

% \paragraph{Proof Technique}
% Use moment generating functions and Markov's inequality.

% \paragraph{Why Chernoff is Powerful}
% \begin{itemize}[noitemsep]
%     \item Exponential decay in deviation
%     \item Much stronger than Chebyshev ($1/k^2$ decay)
%     \item Applies to sums of independent random variables
% \end{itemize}

% \paragraph{Applications}
% \begin{itemize}[noitemsep]
%     \item Randomized rounding in approximation algorithms
%     \item Load balancing with multiple choices
%     \item Probabilistic data structures (Bloom filters)
% \end{itemize}

% \subsubsection{Hoeffding's Inequality}
% \paragraph{For Bounded Random Variables}

% \subsubsection{Azuma's Inequality}
% \paragraph{For Martingales}

% %----------------------------------------------
% \subsection{Randomized Algorithms: Analysis Techniques}
% \label{subsec:randomized-techniques}

% \subsubsection{Las Vegas vs. Monte Carlo}
% \paragraph{Las Vegas}
% \begin{itemize}[noitemsep]
%     \item Always correct
%     \item Random running time
%     \item Example: Randomized QuickSort
% \end{itemize}

% \paragraph{Monte Carlo}
% \begin{itemize}[noitemsep]
%     \item May be incorrect with small probability
%     \item Deterministic running time
%     \item Example: Miller-Rabin primality test
% \end{itemize}

% \subsubsection{Probabilistic Method}
% \paragraph{Erdős's Technique}
% \begin{enumerate}[noitemsep]
%     \item Show expected value has certain property
%     \item Conclude object with that property must exist
% \end{enumerate}

% \paragraph{Example: Graph Coloring Lower Bound}

% \subsubsection{Derandomization Techniques}
% \paragraph{Method of Conditional Expectations}
% \paragraph{Pairwise Independence}
% \paragraph{Limited Independence}

% %----------------------------------------------
% \subsection{Limit Theorems}
% \label{subsec:limit-theorems}

% \subsubsection{Law of Large Numbers}
% \paragraph{Weak Law}
% $\Prob\left(\left|\frac{X_1 + \cdots + X_n}{n} - \mu\right| > \varepsilon\right) \to 0 \text{ as } n \to \infty$

% \paragraph{Strong Law}
% $\Prob\left(\lim_{n \to \infty} \frac{X_1 + \cdots + X_n}{n} = \mu\right) = 1$

% \subsubsection{Central Limit Theorem}
% \paragraph{Statement}
% Let $X_1, X_2, \ldots$ be i.i.d. with $\Expect[X_i] = \mu$, $\Var(X_i) = \sigma^2$. Then:
% $\frac{X_1 + \cdots + X_n - n\mu}{\sigma\sqrt{n}} \xrightarrow{d} \mathcal{N}(0,1)$

% \paragraph{Interpretation}
% Sums of many independent r.v.s are approximately normal.

% \paragraph{Berry-Esseen Theorem}
% Quantifies convergence rate.

% \paragraph{Applications}
% \begin{itemize}[noitemsep]
%     \item Approximating binomial with normal
%     \item Confidence intervals
%     \item Statistical testing
% \end{itemize}

% \subsection{Exercises}
% \subsubsection{Warmup Problems (30)}
% \subsubsection{Standard Problems (40)}
% \subsubsection{Challenging Problems (35)}
% \subsubsection{Research Problems ($\star$) (15)}

% %==============================================
% % SECTION 3: INFORMATION THEORY
% %==============================================
% \section{Information Theory and Entropy}
% \label{sec:information-theory}

% \begin{sectionintro}
% Information theory provides fundamental limits on compression, communication, 
% and computation. Entropy appears in lower bounds for comparison-based sorting, 
% data structure space usage, and communication complexity.

% \textbf{Focus:} We develop Shannon entropy, coding theorems, and applications 
% to algorithm analysis.
% \end{sectionintro}

% %----------------------------------------------
% \subsection{Entropy and Information Content}
% \label{subsec:entropy}

% \subsubsection{Self-Information}
% \paragraph{Definition}
% For event $A$ with $\Prob(A) = p$:
% $I(A) = \log_2 \frac{1}{p} = -\log_2 p \text{ bits}$

% \paragraph{Interpretation}
% Information gained by observing event $A$.

% \subsubsection{Shannon Entropy}
% \paragraph{Definition}
% For discrete random variable $X$ with PMF $p_X$:
% $H(X) = -\sum_{x} p_X(x) \log_2 p_X(x)$

% \paragraph{Alternative Notation}
% $H(X) = \Expect[-\log_2 p_X(X)]$

% \paragraph{Properties}
% \begin{itemize}[noitemsep]
%     \item $H(X) \geq 0$ with equality iff $X$ is deterministic
%     \item $H(X) \leq \log_2 |X|$ with equality iff $X$ is uniform
%     \item Concave function of the distribution
% \end{itemize}

% \subsubsection{Joint and Conditional Entropy}
% \paragraph{Joint Entropy}
% $H(X,Y) = -\sum_{x,y} p_{XY}(x,y) \log_2 p_{XY}(x,y)$

% \paragraph{Conditional Entropy}
% $H(Y|X) = \sum_x p_X(x) H(Y|X=x)$

% \paragraph{Chain Rule}
% $H(X,Y) = H(X) + H(Y|X)$

% \subsubsection{Mutual Information}
% \paragraph{Definition}
% $I(X;Y) = H(X) + H(Y) - H(X,Y) = H(X) - H(X|Y)$

% \paragraph{Interpretation}
% Amount of information $X$ provides about $Y$.

% \paragraph{Properties}
% \begin{itemize}[noitemsep]
%     \item $I(X;Y) \geq 0$ with equality iff $X$ and $Y$ are independent
%     \item Symmetric: $I(X;Y) = I(Y;X)$
% \end{itemize}

% %----------------------------------------------
% \subsection{Source Coding Theorems}
% \label{subsec:source-coding}

% \subsubsection{Prefix-Free Codes}
% \paragraph{Kraft's Inequality}
% For prefix-free code with codeword lengths $\ell_1, \ldots, \ell_n$:
% $\sum_{i=1}^{n} 2^{-\ell_i} \leq 1$

% \paragraph{Converse}
% If $\sum 2^{-\ell_i} \leq 1$, a prefix-free code with these lengths exists.

% \subsubsection{Shannon's Source Coding Theorem}
% \paragraph{Statement}
% For source $X$ with entropy $H(X)$, the expected codeword length $L$ satisfies:
% $H(X) \leq L < H(X) + 1$

% \paragraph{Interpretation}
% Cannot compress below entropy on average.

% \paragraph{Huffman Coding}
% Optimal prefix-free code achieving $L < H(X) + 1$.

% \subsubsection{Arithmetic Coding}
% \paragraph{Achieves $L \approx H(X)$ for long sequences}

% %----------------------------------------------
% \subsection{Applications to Algorithm Analysis}
% \label{subsec:info-theory-algorithms}

% \subsubsection{Sorting Lower Bound}
% \paragraph{Information-Theoretic Argument}
% \begin{itemize}[noitemsep]
%     \item $n!$ possible permutations
%     \item Each comparison provides $\leq 1$ bit of information
%     \item Need $\log_2(n!)$ bits total
%     \item By Stirling: $\log_2(n!) \approx n \log_2 n$
% \end{itemize}

% Therefore: comparison sorting requires $\Omega(n \log n)$ comparisons.

% \subsubsection{Lower Bounds for Data Structures}
% \paragraph{Cell-Probe Model}
% \paragraph{Information Transfer}

% \subsubsection{Communication Complexity}
% \paragraph{Two-Party Communication}
% \paragraph{Equality Testing Lower Bound}

% \subsection{Exercises}
% \subsubsection{Warmup Problems (20)}
% \subsubsection{Standard Problems (25)}
% \subsubsection{Challenging Problems (20)}
% \subsubsection{Research Problems ($\star$) (10)}

% %==============================================
% % SECTION 4: NUMBER THEORY
% %==============================================
% \section{Number Theory for Algorithmic Applications}
% \label{sec:number-theory}

% \begin{sectionintro}
% Number theory is essential for cryptography, hashing, pseudorandom generation, 
% and complexity theory. This section develops computational number theory with 
% complete proofs and complexity analysis.

% \textbf{Unique Feature:} We analyze the complexity of every algorithm we present.
% \end{sectionintro}

% %----------------------------------------------
% \subsection{Divisibility and Modular Arithmetic}
% \label{subsec:modular-arithmetic}

% \subsubsection{Division Algorithm}
% \paragraph{Theorem}
% For integers $a$ and $b > 0$, there exist unique integers $q$ and $r$ such that:
% $a = bq + r, \quad 0 \leq r < b$

% \paragraph{Proof}
% \begin{proof}
% Existence: Consider $S = \{a - kb : k \in \mathbb{Z}, a - kb \geq 0\}$. 
% Let $r$ be the minimum element. Then $r = a - qb$ for some $q$. If $r \geq b$, 
% then $r - b \in S$ contradicts minimality.

% Uniqueness: If $a = bq_1 + r_1 = bq_2 + r_2$, then $b(q_1 - q_2) = r_2 - r_1$. 
% Since $|r_2 - r_1| < b$, we must have $q_1 = q_2$ and $r_1 = r_2$.
% \end{proof}

% \subsubsection{Congruences}
% \paragraph{Definition}
% $a \equiv b \pmod{m}$ iff $m \mid (a - b)$

% \paragraph{Properties}
% \begin{itemize}[noitemsep]
%     \item Reflexive, symmetric, transitive (equivalence relation)
%     \item Compatible with addition: $a \equiv b \pmod{m} \implies a+c \equiv b+c \pmod{m}$
%     \item Compatible with multiplication: $a \equiv b \pmod{m} \implies ac \equiv bc \pmod{m}$
% \end{itemize}

% \subsubsection{Modular Arithmetic}
% \paragraph{$\mathbb{Z}_m = \{0, 1, \ldots, m-1\}$}
% \paragraph{Operations: $+_m, \times_m$}
% \paragraph{Complexity: $O(\log m)$ per operation (using long arithmetic)}

% \subsubsection{Units and Inverses}
% \paragraph{Definition}
% $a \in \mathbb{Z}_m$ is a unit if $\exists b : ab \equiv 1 \pmod{m}$

% \paragraph{Theorem}
% $a$ is a unit in $\mathbb{Z}_m$ iff $\gcd(a, m) = 1$.

% \paragraph{Euler's Totient Function}
% $\phi(m) = |\{a \in \mathbb{Z}_m : \gcd(a,m) = 1\}|$

% \paragraph{Computing $\phi(m)$}
% If $m = p_1^{a_1} \cdots p_k^{a_k}$:
% $\phi(m) = m \prod_{i=1}^{k} \left(1 - \frac{1}{p_i}\right)$

% %----------------------------------------------
% \subsection{Greatest Common Divisor}
% \label{subsec:gcd}

% \subsubsection{Definition and Properties}
% \paragraph{$\gcd(a,b)$ is the largest positive integer dividing both $a$ and $b$}

% \paragraph{Properties}
% \begin{itemize}[noitemsep]
%     \item $\gcd(a,b) = \gcd(b,a)$
%     \item $\gcd(a,0) = |a|$
%     \item $\gcd(a,b) = \gcd(a-b, b)$
% \end{itemize}

% \subsubsection{Euclidean Algorithm}
% \paragraph{Algorithm}
% \begin{algorithm}[H]
% \caption{Euclidean Algorithm}
% \begin{algorithmic}[1]
% \Require Integers $a, b$ with $a \geq b > 0$
% \Ensure $\gcd(a,b)$
% \While{$b \neq 0$}
%     \State $r \gets a \bmod b$
%     \State $a \gets b$
%     \State $b \gets r$
% \EndWhile
% \State \Return $a$
% \end{algorithmic}
% \end{algorithm}

% \paragraph{Correctness}
% Invariant: $\gcd(a,b)$ remains constant throughout.

% \paragraph{Complexity Analysis}
% \begin{theorem}[Lamé's Theorem]
% The number of steps in Euclidean algorithm for $\gcd(a,b)$ with $a \geq b$ is 
% $O(\log b)$.
% \end{theorem}

% \begin{proof}
% If $a = bq + r$ with $r < b/2$, then two steps reduce the larger number by at 
% least half. If $r \geq b/2$, the next step gives $b = r \cdot 1 + (b-r)$ with 
% $b - r < b/2$. So every two steps reduce by half, giving $O(\log b)$ steps.
% \end{proof}

% \paragraph{Worst Case: Consecutive Fibonacci Numbers}
% $\gcd(F_{n+1}, F_n) \text{ requires exactly } n-1 \text{ steps}$

% \subsubsection{Extended Euclidean Algorithm}
% \paragraph{Goal}
% Find integers $x, y$ such that $ax + by = \gcd(a,b)$ (Bézout's identity).

% \paragraph{Algorithm}
% \begin{algorithm}[H]
% \caption{Extended Euclidean Algorithm}
% \begin{algorithmic}[1]
% \Require Integers $a, b$
% \Ensure $(\gcd(a,b), x, y)$ where $ax + by = \gcd(a,b)$
% \If{$b = 0$}
%     \State \Return $(a, 1, 0)$
% \EndIf
% \State $(d, x', y') \gets \text{ExtendedGCD}(b, a \bmod b)$
% \State $x \gets y'$
% \State $y \gets x' - \lfloor a/b \rfloor \cdot y'$
% \State \Return $(d, x, y)$
% \end{algorithmic}
% \end{algorithm}

% \paragraph{Applications}
% \begin{itemize}[noitemsep]
%     \item Computing modular inverses
%     \item Solving linear Diophantine equations
%     \item Chinese Remainder Theorem
% \end{itemize}

% %----------------------------------------------
% \subsection{Prime Numbers}
% \label{subsec:primes}

% \subsubsection{Definition and Basic Properties}
% \paragraph{Prime: $p > 1$ with only divisors 1 and $p$}
% \paragraph{Composite: $n > 1$ that is not prime}

% \subsubsection{Fundamental Theorem of Arithmetic}
% \paragraph{Statement}
% Every integer $n > 1$ can be written uniquely (up to order) as:
% $n = p_1^{a_1} p_2^{a_2} \cdots p_k^{a_k}$
% where $p_i$ are distinct primes and $a_i > 0$.

% \paragraph{Proof}
% Existence by strong induction; uniqueness by Euclid's lemma.

% \subsubsection{Infinitude of Primes}
% \paragraph{Euclid's Proof}
% \begin{proof}
% Suppose finitely many primes $p_1, \ldots, p_k$. Consider:
% $N = p_1 \cdots p_k + 1$
% $N$ is not divisible by any $p_i$, so either $N$ is prime (contradiction) or 
% $N$ has a prime divisor not in the list (contradiction).
% \end{proof}

% \paragraph{Euler's Proof (via $\zeta$ function)}

% \subsubsection{Prime Counting Function}
% \paragraph{Definition: $\pi(x) = |\{p \leq x : p \text{ prime}\}|$}

% \paragraph{Prime Number Theorem}
% $\pi(x) \sim \frac{x}{\ln x} \text{ as } x \to \infty$

% \paragraph{Better Approximation}
% $\pi(x) \sim \text{Li}(x) = \int_2^x \frac{dt}{\ln t}$

% \subsubsection{Sieve of Eratosthenes}
% \paragraph{Algorithm}
% \begin{enumerate}[noitemsep]
%     \item Create list of integers from 2 to $n$
%     \item For each prime $p \leq \sqrt{n}$: mark multiples $2p, 3p, \ldots$
%     \item Remaining unmarked numbers are prime
% \end{enumerate}

% \paragraph{Complexity: $O(n \log \log n)$}

% \paragraph{Space: $O(n)$ bits using bit array}

% \subsubsection{Primality Testing}
% \paragraph{Trial Division: $O(\sqrt{n})$ time}

% \paragraph{Fermat's Little Theorem}
% If $p$ is prime and $\gcd(a,p) = 1$:
% $a^{p-1} \equiv 1 \pmod{p}$

% \paragraph{Miller-Rabin Test}
% \begin{itemize}[noitemsep]
%     \item Probabilistic algorithm
%     \item Error probability $\leq 4^{-k}$ with $k$ rounds
%     \item Complexity: $O(k \log^3 n)$ using fast modular exponentiation
% \end{itemize}

% \paragraph{AKS Primality Test}
% \begin{itemize}[noitemsep]
%     \item Deterministic polynomial-time algorithm
%     \item Complexity: $\tilde{O}(\log^{6} n)$ (original), improved to $\tilde{O}(\log^{6} n)$
%     \item Theoretical importance > practical use
% \end{itemize}

% %----------------------------------------------
% \subsection{Modular Exponentiation}
% \label{subsec:modexp}

% \subsubsection{Repeated Squaring Algorithm}
% \paragraph{Goal: Compute $a^n \bmod m$ efficiently}

% \paragraph{Algorithm}
% \begin{algorithm}[H]
% \caption{Modular Exponentiation}
% \begin{algorithmic}[1]
% \Require $a, n, m$ with $n \geq 0$, $m > 0$
% \Ensure $a^n \bmod m$
% \State $result \gets 1$
% \State $base \gets a \bmod m$
% \While{$n > 0$}
%     \If{$n \bmod 2 = 1$}
%         \State $result \gets (result \times base) \bmod m$
%     \EndIf
%     \State $n \gets \lfloor n/2 \rfloor$
%     \State $base \gets (base \times base) \bmod m$
% \EndWhile
% \State \Return $result$
% \end{algorithmic}
% \end{algorithm}

% \paragraph{Complexity: $O(\log n)$ multiplications mod $m$}
% \paragraph{Total: $O(\log n \cdot \log^2 m)$ bit operations}

% %----------------------------------------------
% \subsection{Chinese Remainder Theorem}
% \label{subsec:crt}

% \subsubsection{Statement}
% \paragraph{Theorem}
% Let $m_1, \ldots, m_k$ be pairwise coprime positive integers. Then the system:
% \begin{align*}
% x &\equiv a_1 \pmod{m_1} \\
% &\vdots \\
% x &\equiv a_k \pmod{m_k}
% \end{align*}
% has a unique solution modulo $M = m_1 \cdots m_k$.

% \subsubsection{Constructive Proof and Algorithm}
% \paragraph{Let $M_i = M / m_i$}
% \paragraph{Find $y_i$ such that $M_i y_i \equiv 1 \pmod{m_i}$ (using Extended GCD)}
% \paragraph{Solution: $x = \sum_{i=1}^{k} a_i M_i y_i \pmod{M}$}

% \subsubsection{Applications}
% \paragraph{Fast Modular Arithmetic}
% \paragraph{RSA Speedup}
% \paragraph{Secret Sharing Schemes}

% %----------------------------------------------
% \subsection{Applications to Cryptography}
% \label{subsec:cryptography}

% \subsubsection{RSA Cryptosystem}
% \paragraph{Key Generation}
% \begin{enumerate}[noitemsep]
%     \item Choose large primes $p, q$
%     \item $n = pq$, $\phi(n) = (p-1)(q-1)$
%     \item Choose $e$ with $\gcd(e, \phi(n)) = 1$
%     \item Compute $d \equiv e^{-1} \pmod{\phi(n)}$
%     \item Public key: $(n, e)$; private key: $(n, d)$
% \end{enumerate}

% \paragraph{Encryption/Decryption}
% \begin{itemize}[noitemsep]
%     \item Encrypt: $c \equiv m^e \pmod{n}$
%     \item Decrypt: $m \equiv c^d \pmod{n}$
% \end{itemize}

% \paragraph{Correctness: Euler's Theorem}
% $m^{\phi(n)} \equiv 1 \pmod{n} \implies m^{ed} \equiv m \pmod{n}$

% \paragraph{Security}
% Based on hardness of integer factorization.

% \subsubsection{Diffie-Hellman Key Exchange}
% \paragraph{Protocol}
% \paragraph{Security: Discrete Logarithm Problem}

% \subsubsection{Hash Functions}
% \paragraph{Universal Hashing}
% \paragraph{Cryptographic Hash Functions: SHA-256, SHA-3}

% \subsection{Exercises}
% \subsubsection{Warmup Problems (25)}
% \subsubsection{Standard Problems (30)}
% \subsubsection{Challenging Problems (25)}
% \subsubsection{Research Problems ($\star$) (10)}

% %==============================================
% % CHAPTER SUMMARY
% %==============================================
% \section{Chapter Summary and Road Ahead}
% \label{sec:chapter2-summary}

% \begin{sectionintro}
% This chapter developed the analytical and probabilistic foundations for 
% algorithm analysis. We've covered:

% \begin{itemize}[noitemsep]
%     \item \textbf{Analysis:} Limits, series, asymptotic methods, Stirling's approximation
%     \item \textbf{Probability:} Rigorous foundations, concentration inequalities, limit theorems
%     \item \textbf{Information Theory:} Entropy, coding theorems, lower bounds
%     \item \textbf{Number Theory:} GCD algorithms, primality testing, modular arithmetic
% \end{itemize}

% \textbf{What's Next:}

% Part 2 begins the formal study of algorithm analysis, building on these 
% mathematical foundations:
% \begin{itemize}[noitemsep]
%     \item Rigorous asymptotic notation
%     \item Recurrence relation solving methods
%     \item Probabilistic analysis of randomized algorithms
%     \item Amortized analysis techniques
% \end{itemize}
% \end{sectionintro}

% %==============================================
% % END OF CHAPTER 2
% %==============================================] = \sum_x x \cdot p_X(x)$$

% \paragraph{Law of the Unconscious Statistician}
% $$\Expect[g(X)] = \sum_x g(x) \cdot p_X(x)$$

% \paragraph{Properties}
% \begin{itemize}[noitemsep]
%     \item Linearity: $\Expect[aX + bY] = a\Expect[X] + b\Expect[Y]$
%     \item Monotonicity: $X \leq Y \implies \Expect[X] \leq \Expect[Y]$
% \end{itemize}

% \subsubsection{Variance and Standard Deviation}
% \paragraph{Variance}
% $$\Var(X) = \Expect[(X - \Expect[X])^2] = \Expect[X^2] - (\Expect[X])^2$$

% \paragraph{Standard Deviation}
% $$\sigma_X = \sqrt{\Var(X)}$$

% \paragraph{Properties}
% \begin{itemize}[noitemsep]
%     \item $\Var(aX + b) = a^2 \Var(X)$
%     \item If $X, Y$ independent: $\Var(X + Y) = \Var(X) + \Var(Y)$
% \end{itemize}

% \subsubsection{Higher Moments}
% \paragraph{$k$-th Moment: $\Expect[X^k]$}
% \paragraph{Moment Generating Functions}

% %----------------------------------------------
% \subsection{Common Probability Distributions}
% \label{subsec:distributions-detailed}

% \subsubsection{Bernoulli Distribution}
% \paragraph{Parameters: $p \in [0,1]$}
% \paragraph{PMF: $\Prob(X = 1) = p, \Prob(X = 0) = 1-p$}
% \paragraph{$\Expect[X] = p, \Var(X) = p(1-p)$}

% \subsubsection{Binomial Distribution}
% \paragraph{Parameters: $n \in \mathbb{N}, p \in [0,1]$}
% \paragraph{PMF: $\Prob(X = k) = \binom{n}{k} p^k (1-p)^{n-k}$}
% \paragraph{$\Expect[X] = np, \Var(X) = np(1-p)$}
% \paragraph{Applications: Success counts in $n$ trials}

% \subsubsection{Geometric Distribution}
% \paragraph{Parameters: $p \in (0,1]$}
% \paragraph{PMF: $\Prob(X = k) = (1-p)^{k-1} p$}
% \paragraph{$\Expect[X] = 1/p, \Var(X) = (1-p)/p^2$}
% \paragraph{Memoryless Property}
% \paragraph{Applications: Waiting times}

% \subsubsection{Poisson Distribution}
% \paragraph{Parameters: $\lambda > 0$}
% \paragraph{PMF: $\Prob(X = k) = \frac{\lambda^k e^{-\lambda}}{k!}$}
% \paragraph{$\Expect[X] = \lambda, \Var(X) = \lambda$}
% \paragraph{Approximation to Binomial}
% \paragraph{Applications: Rare events, hashing}

% \subsubsection{Uniform Distribution}
% \paragraph{Discrete: $X \in \{1, 2, \ldots, n\}$ equally likely}
% \paragraph{$\Expect[X] = (n+1)/2, \Var(X) = (n^2-1)/12$}

% \subsubsection{Negative Binomial Distribution}
% \paragraph{Waiting for $r$ successes}

% %----------------------------------------------
% \subsection{Linearity of Expectation}
% \label{subsec:linearity-detailed}

% \begin{subsectionintro}
% Linearity of expectation is the single most powerful tool in probabilistic 
% algorithm analysis. We prove it rigorously and demonstrate its power.
% \end{subsectionintro}

% \subsubsection{Statement and Proof}
% \paragraph{Theorem}
% For any random variables $X_1, \ldots, X_n$ (not necessarily independent):
% $$\Expect\left[\sum_{i=1}^{n} X_i\right] = \sum_{i=1}^{n} \Expect[X_i]$$

% \paragraph{Proof}
% \begin{proof}
% Let $X = \sum_{i=1}^{n} X_i$. Then:
% \begin{align*}
% \Expect[X] &= \sum_{(x_1, \ldots, x_n)} \left(\sum_{i=1}^{n} x_i\right) \Prob(X_1=x_1, \ldots, X_n=x_n) \\
% &= \sum_{i=1}^{n} \sum_{(x_1, \ldots, x_n)} x_i \Prob(X_1=x_1, \ldots, X_n=x_n) \\
% &= \sum_{i=1}^{n} \Expect[X_i]
% \end{align*}
% \end{proof}

% \subsubsection{Why Independence Is Not Required}
% \paragraph{Key Insight}
% The proof uses only the definition of expectation and the distributive law—no 
% independence assumption needed.

% \subsubsection{Indicator Random Variables}
% \paragraph{Definition}
% $$I_A = \begin{cases} 1 & \text{if event } A \text{ occurs} \\ 0 & \text{otherwise} \end{cases}$$

% \paragraph{Key Property}
% $$\Expect[I_A] = \Prob(A)$$

% \paragraph{Power of Indicators}
% Express complicated random variables as sums of indicators.

% \subsubsection{Case Study: QuickSort Analysis}
% \paragraph{Setup}
% Let $X$ = number of comparisons in QuickSort on $n$ elements.

% \paragraph{Indicator Variable Approach}
% Define $X_{ij} = 1$ if elements $i$ and $j$ are compared, 0 otherwise.

% Then: $X = \sum_{1 \leq i < j \leq n} X_{ij}$

% \paragraph{Computing $\Expect[X_{ij}]$}
% Elements $i$ and $j$ are compared iff one is chosen as pivot before any element 
% between them. Probability: $\frac{2}{j-i+1}$.

% \paragraph{Final Calculation}
% \begin{align*}
% \Expect[X] &= \sum_{1 \leq i < j \leq n} \Expect[X_{ij}] \\
% &= \sum_{1 \leq i < j \leq n} \frac{2}{j-i+1} \\
% &= \sum_{k=1}^{n-1} (n-k) \cdot \frac{2}{k+1} \\
% &= 2(n+1) \sum_{k=1}^{n} \frac{1}{k} - 2n \\
% &= 2(n+1) H_n - 2n \\
% &\sim 2n \ln n
% \end{align*}

% Thus $\Expect[X] = \Theta(n \log n)$.

% \subsubsection{More Applications}
% \paragraph{Expected Number of Fixed Points in Random Permutation}
% \paragraph{Coupon Collector Problem}
% \paragraph{Load Balancing: Balls into Bins}

% %----------------------------------------------
% \subsection{Concentration Inequalities}
% \label{subsec:concentration}

% \begin{subsectionintro}
% Knowing the expected value is not enough—we need bounds on how far a random 
% variable can deviate from its expectation.
% \end{subsectionintro}

% \subsubsection{Markov's Inequality}
% \paragraph{Statement}
% For any non-negative random variable $X$ and $a > 0$:
% $$\Prob(X \geq a) \leq \frac{\Expect[X]}{a}$$

% \paragraph{Proof}
% \begin{proof}
% Let $I = \mathbbm{1}_{X \geq a}$. Then $X \geq a \cdot I$, so:
% $$\Expect[X
\chapter{Agricultural Complexity and Token Systems}
\chapter{The Birth of Written Mathematics}