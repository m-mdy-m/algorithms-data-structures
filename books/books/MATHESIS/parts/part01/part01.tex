\part{Origins of Mathematical Thought}
\label{part:origins}

\begin{partintro}
\lettrine[lines=3]{M}{athematics did not} emerge fully formed from human minds. It was forged through millennia of necessity, observation, and intellectual struggle. Before symbols existed, before numbers had names, our ancestors confronted the fundamental challenge: how to comprehend and communicate quantity, pattern, and structure.

This part traces mathematics from its primordial origins—when humanity first distinguished "one" from "many"—through the revolutionary abstractions that made systematic thought possible. We examine not merely what ancient peoples calculated, but how they reasoned, what cognitive leaps enabled mathematical thinking, and why certain cultures developed particular mathematical frameworks.

\vspace{1em}
\textbf{What Makes This Different:}
\begin{itemize}[noitemsep]
    \item \textbf{Cognitive Foundations:} We explore the neurological and psychological basis for mathematical intuition
    \item \textbf{Archaeological Evidence:} Physical artifacts reveal how abstract concepts became material reality
    \item \textbf{Cultural Contexts:} Mathematical systems emerged from specific human needs and worldviews
    \item \textbf{Conceptual Evolution:} We trace how simple counting became sophisticated abstraction
\end{itemize}

\begin{quote}
\textit{``The numbers are a match for the transcendent world, and the transcendent world is a match for the numbers.''}

\hfill--- \textsc{Aristotle, Metaphysics}
\end{quote}
\end{partintro}

\chapter{Mathematical Foundations I: Discrete Structures}
\label{ch:discrete-foundations}

\lettrine[lines=3]{D}{iscrete mathematics} is the native language of computer 
science. Unlike the continuous mathematics of physics and engineering, we work 
with countable, distinct objects: integers, graphs, logic formulas, finite 
automata. This chapter develops the discrete structures essential for 
algorithmic reasoning.

\begin{chapterintro}
\textbf{Chapter Organization:}

This chapter covers five fundamental areas:
\begin{enumerate}[noitemsep]
    \item \textbf{Logic and Proof Theory} — The language of correctness
    \item \textbf{Set Theory and Relations} — The foundation of abstraction
    \item \textbf{Combinatorial Analysis} — Counting and enumeration
    \item \textbf{Graph Theory} — Modeling relationships
    \item \textbf{Algebraic Structures} — Symmetry and structure
\end{enumerate}

Each section includes:
\begin{itemize}[noitemsep]
    \item Complete proofs of major theorems
    \item Worked examples from algorithm analysis
    \item Historical context and development
    \item Exercises at three levels: Basic, Intermediate, Advanced
    \item Research-level problems marked with $\star$
\end{itemize}

\textbf{Prerequisites:} Mathematical maturity at the level of a first-year 
university mathematics course. Comfort with symbolic manipulation and logical 
reasoning.
\end{chapterintro}

%----------------------------------------------
% SECTION 1: LOGIC AND PROOF THEORY
%----------------------------------------------
\section{Logic and Proof Theory}
\label{sec:logic-proof}

\begin{sectionintro}
Before we can prove algorithms correct or establish complexity bounds, we must 
understand the nature of proof itself. This section develops propositional and 
predicate logic, formal proof systems, and the art of mathematical argumentation.

\textbf{Why This Matters:} Every correctness proof, every lower bound, every 
impossibility result rests on logical foundations. Mastery of proof techniques 
is not optional—it is the essence of computer science.
\end{sectionintro}

\subsection{Propositional Logic}
\label{subsec:propositional-logic}

\subsubsection{Syntax: Well-Formed Formulas}
\paragraph{Formal Definition.}
\paragraph{Parse Trees and Unique Readability.}
\paragraph{Inductive Definition of Formulas.}

\subsubsection{Semantics: Truth Tables and Valuations}
\paragraph{Boolean Valuations.}
\paragraph{Truth-Functional Connectives: $\land, \lor, \neg, \rightarrow, \leftrightarrow$.}
\paragraph{Tautologies, Contradictions, and Contingencies.}

\subsubsection{Logical Equivalence and Normal Forms}
\paragraph{Substitution and Equivalence Relations.}
\paragraph{Conjunctive Normal Form (CNF).}
\paragraph{Disjunctive Normal Form (DNF).}
\paragraph{Theorem: Every Formula Has Equivalent CNF/DNF.}

\subsubsection{Proof Systems for Propositional Logic}
\paragraph{Natural Deduction.}
\paragraph{Sequent Calculus.}
\paragraph{Resolution Principle and SAT Solving.}

\subsection{Predicate Logic (First-Order Logic)}
\label{subsec:predicate-logic}

\subsubsection{Syntax: Terms, Predicates, Quantifiers}
\paragraph{Variables, Constants, Function Symbols.}
\paragraph{Universal ($\forall$) and Existential ($\exists$) Quantifiers.}
\paragraph{Free vs. Bound Variables.}

\subsubsection{Semantics: Structures and Models}
\paragraph{Interpretations and Domains.}
\paragraph{Satisfaction and Truth in a Model.}
\paragraph{Validity and Logical Consequence.}

\subsubsection{Prenex Normal Form}
\paragraph{Quantifier Scope and Movement Rules.}
\paragraph{Skolemization and Herbrand's Theorem.}

\subsubsection{Decidability and Undecidability}
\paragraph{Church-Turing Theorem: First-Order Logic is Undecidable.}
\paragraph{Decidable Fragments and Applications to Verification.}

\subsection{Proof Techniques and Strategies}
\label{subsec:proof-techniques}

\subsubsection{Direct Proof}
\paragraph{Structure and Template.}
\paragraph{Example: Properties of Even/Odd Integers.}

\subsubsection{Proof by Contradiction (Reductio ad Absurdum)}
\paragraph{Strategy and When to Use.}
\paragraph{Example: $\sqrt{2}$ is Irrational.}
\paragraph{Example: Infinitude of Primes (Euclid).}

\subsubsection{Proof by Contrapositive}
\paragraph{Logical Equivalence to Direct Proof.}
\paragraph{When Contrapositive is Easier.}

\subsubsection{Mathematical Induction}
\paragraph{Weak Induction Principle.}
\paragraph{Strong Induction (Complete Induction).}
\paragraph{Well-Ordering Principle and Equivalence.}
\paragraph{Structural Induction on Recursive Data Types.}
\paragraph{Example: Correctness of Merge Sort.}

\subsubsection{Proof by Construction}
\paragraph{Existence Proofs: Constructive vs. Non-Constructive.}
\paragraph{Algorithmic Content of Constructive Proofs.}

\subsubsection{Proof by Counterexample}
\paragraph{Disproving Universal Statements.}
\paragraph{Minimal Counterexamples.}

\subsubsection{Pigeonhole Principle}
\paragraph{Simple Form: $n+1$ Objects in $n$ Boxes.}
\paragraph{Generalized Form: Distribution Arguments.}
\paragraph{Example: Birthday Paradox.}
\paragraph{Example: Lower Bounds for Sorting.}

\subsection{Common Proof Errors and How to Avoid Them}
\label{subsec:proof-errors}

\subsubsection{Circular Reasoning (Begging the Question)}
\subsubsection{Proof by Example}
\subsubsection{Confusion of Necessary and Sufficient Conditions}
\subsubsection{Quantifier Errors}
\subsubsection{Vacuous Truth}

\subsection{Advanced Topics in Logic}
\label{subsec:advanced-logic}

\subsubsection{Temporal Logic and Program Verification}
\subsubsection{Modal Logic and Epistemic Logic}
\subsubsection{Gödel's Incompleteness Theorems (Informal)}

\subsection{Exercises}
\label{subsec:logic-exercises}

\subsubsection{Warmup Problems (20 problems)}
\subsubsection{Standard Problems (30 problems)}
\subsubsection{Challenging Problems (25 problems)}
\subsubsection{Research-Level Problems ($\star$) (10 problems)}

%----------------------------------------------
% SECTION 2: SET THEORY AND RELATIONS
%----------------------------------------------
\section{Set Theory, Functions, and Relations}
\label{sec:set-theory}

\begin{sectionintro}
Sets are the foundation of modern mathematics. Functions and relations provide 
the vocabulary for describing computations and data structures. This section 
develops these concepts rigorously, with emphasis on applications to algorithm 
design.
\end{sectionintro}

\subsection{Axiomatic Set Theory (Informal)}
\label{subsec:axiomatic-sets}

\subsubsection{Zermelo-Fraenkel Axioms (Intuitive Overview)}
\subsubsection{Russell's Paradox and the Need for Axioms}
\subsubsection{The Axiom of Choice and Its Consequences}

\subsection{Basic Set Operations}
\label{subsec:set-operations}

\subsubsection{Set Builder Notation and Comprehension}
\subsubsection{Union, Intersection, Difference, Symmetric Difference}
\subsubsection{Power Set and Cartesian Product}
\subsubsection{De Morgan's Laws and Duality}

\subsection{Cardinality and Counting}
\label{subsec:cardinality}

\subsubsection{Finite, Countable, and Uncountable Sets}
\subsubsection{Cantor's Diagonal Argument}
\subsubsection{Schröder-Bernstein Theorem}
\subsubsection{Continuum Hypothesis}

\subsection{Functions: Mappings Between Sets}
\label{subsec:functions}

\subsubsection{Definition and Notation}
\paragraph{Domain, Codomain, Range.}
\paragraph{Function Composition and Associativity.}

\subsubsection{Types of Functions}
\paragraph{Injective (One-to-One) Functions.}
\paragraph{Surjective (Onto) Functions.}
\paragraph{Bijective Functions and Inverses.}
\paragraph{Theorem: Composition of Bijections is Bijective.}

\subsubsection{Special Functions}
\paragraph{Identity Function.}
\paragraph{Constant Functions.}
\paragraph{Projection Functions.}

\subsubsection{Image and Preimage}
\paragraph{Properties of Image and Preimage.}
\paragraph{Inverse Image and Set Operations.}

\subsection{Relations}
\label{subsec:relations}

\subsubsection{Binary Relations}
\paragraph{Definition as Subsets of Cartesian Products.}
\paragraph{Representation: Matrices, Graphs, Tables.}

\subsubsection{Properties of Relations}
\paragraph{Reflexivity, Irreflexivity.}
\paragraph{Symmetry, Antisymmetry, Asymmetry.}
\paragraph{Transitivity.}

\subsubsection{Equivalence Relations}
\paragraph{Definition and Examples.}
\paragraph{Equivalence Classes and Partitions.}
\paragraph{Theorem: Equivalence Relations Induce Partitions.}
\paragraph{Quotient Sets.}

\subsubsection{Partial Orders}
\paragraph{Definition: Reflexive, Antisymmetric, Transitive.}
\paragraph{Hasse Diagrams.}
\paragraph{Maximal and Minimal Elements.}
\paragraph{Least Upper Bound (Supremum) and Greatest Lower Bound (Infimum).}
\paragraph{Lattices.}

\subsubsection{Total Orders (Linear Orders)}
\paragraph{Definition and Examples.}
\paragraph{Well-Orderings.}
\paragraph{Application: Topological Sorting.}

\subsection{Exercises}
\subsubsection{Warmup Problems}
\subsubsection{Standard Problems}
\subsubsection{Challenging Problems}
\subsubsection{Research Problems ($\star$)}

%----------------------------------------------
% SECTION 3: COMBINATORIAL ANALYSIS
%----------------------------------------------
\section{Combinatorial Analysis and Enumeration}
\label{sec:combinatorics}

\begin{sectionintro}
Combinatorics is the mathematics of counting. It answers fundamental questions: 
How many possible inputs exist? How many steps must an algorithm take? What is 
the size of the search space?

This section goes far beyond basic permutations and combinations, developing 
generating functions, recurrence relations, and asymptotic enumeration 
techniques essential for algorithm analysis.
\end{sectionintro}

\subsection{Fundamental Counting Principles}
\label{subsec:counting-principles}

\subsubsection{Sum Rule (Addition Principle)}
\subsubsection{Product Rule (Multiplication Principle)}
\subsubsection{Division Principle (Quotient Principle)}
\subsubsection{Bijection Principle}

\subsection{Permutations and Arrangements}
\label{subsec:permutations}

\subsubsection{Permutations of $n$ Distinct Objects}
\paragraph{Formula: $n!$}
\paragraph{Stirling's Approximation (Preview).}

\subsubsection{Permutations with Repetition}
\paragraph{Multiset Permutations: $\frac{n!}{n_1! n_2! \cdots n_k!}$}

\subsubsection{$k$-Permutations: $P(n,k) = \frac{n!}{(n-k)!}$}

\subsubsection{Circular Permutations}
\paragraph{$(n-1)!$ Arrangements.}

\subsubsection{Derangements}
\paragraph{Permutations with No Fixed Points.}
\paragraph{Formula: $D_n = n! \sum_{i=0}^{n} \frac{(-1)^i}{i!}$}
\paragraph{Asymptotic: $D_n \sim \frac{n!}{e}$}

\subsection{Combinations and Selections}
\label{subsec:combinations}

\subsubsection{Binomial Coefficients: $\binom{n}{k}$}
\paragraph{Definition and Pascal's Triangle.}
\paragraph{Symmetry: $\binom{n}{k} = \binom{n}{n-k}$}
\paragraph{Vandermonde's Identity.}

\subsubsection{Binomial Theorem}
\paragraph{$(x+y)^n = \sum_{k=0}^{n} \binom{n}{k} x^k y^{n-k}$}
\paragraph{Applications to Probability.}

\subsubsection{Multinomial Coefficients}
\paragraph{$\binom{n}{k_1, k_2, \ldots, k_m} = \frac{n!}{k_1! k_2! \cdots k_m!}$}
\paragraph{Multinomial Theorem.}

\subsubsection{Combinations with Repetition}
\paragraph{Stars and Bars Method.}
\paragraph{Formula: $\binom{n+k-1}{k}$}

\subsection{Inclusion-Exclusion Principle}
\label{subsec:inclusion-exclusion}

\subsubsection{Two-Set Formula}
\subsubsection{General Formula for $n$ Sets}
\subsubsection{Applications}
\paragraph{Counting Surjections.}
\paragraph{Euler's Totient Function.}
\paragraph{Derangement Formula (Alternative Derivation).}

\subsection{Pigeonhole Principle and Ramsey Theory}
\label{subsec:pigeonhole}

\subsubsection{Simple Pigeonhole Principle}
\subsubsection{Generalized Pigeonhole Principle}
\subsubsection{Ramsey's Theorem (Introduction)}
\paragraph{$R(3,3) = 6$ — The Party Problem.}
\paragraph{Bounds on $R(m,n)$.}

\subsection{Generating Functions}
\label{subsec:generating-functions}

\begin{subsectionintro}
Generating functions transform combinatorial problems into algebra. They are 
indispensable for solving recurrences and asymptotic enumeration.
\end{subsectionintro}

\subsubsection{Ordinary Generating Functions (OGF)}
\paragraph{Definition: $A(x) = \sum_{n=0}^{\infty} a_n x^n$}
\paragraph{Operations on Generating Functions.}
\paragraph{Convolution and Product of GFs.}

\subsubsection{Exponential Generating Functions (EGF)}
\paragraph{Definition: $A(x) = \sum_{n=0}^{\infty} a_n \frac{x^n}{n!}$}
\paragraph{When to Use EGF vs. OGF.}

\subsubsection{Solving Recurrences with Generating Functions}
\paragraph{Example: Fibonacci Numbers.}
\paragraph{Example: Catalan Numbers.}

\subsubsection{Coefficient Extraction}
\paragraph{Cauchy's Residue Theorem (Overview).}
\paragraph{Asymptotic Analysis via Singularities.}

\subsection{Recurrence Relations}
\label{subsec:recurrences-combinatorial}

\subsubsection{Linear Recurrences with Constant Coefficients}
\paragraph{Homogeneous Recurrences.}
\paragraph{Characteristic Equation Method.}
\paragraph{Example: Fibonacci, Tribonacci.}

\subsubsection{Non-Homogeneous Recurrences}
\paragraph{Method of Undetermined Coefficients.}
\paragraph{Particular Solutions.}

\subsubsection{Divide-and-Conquer Recurrences}
\paragraph{Form: $T(n) = aT(n/b) + f(n)$}
\paragraph{Preview of Master Theorem (Full treatment in Part 2).}

\subsection{Catalan Numbers}
\label{subsec:catalan}

\subsubsection{Definition and Recurrence}
\subsubsection{Closed Form: $C_n = \frac{1}{n+1}\binom{2n}{n}$}
\subsubsection{Applications}
\paragraph{Balanced Parentheses.}
\paragraph{Binary Search Trees.}
\paragraph{Triangulations of Polygons.}

\subsection{Stirling Numbers}
\label{subsec:stirling}

\subsubsection{Stirling Numbers of the First Kind}
\paragraph{Unsigned: Number of Permutations with $k$ Cycles.}
\paragraph{Signed: Coefficients in Falling Factorial Expansion.}

\subsubsection{Stirling Numbers of the Second Kind}
\paragraph{Definition: $S(n,k)$ — Partitions of $n$ Elements into $k$ Subsets.}
\paragraph{Recurrence: $S(n,k) = k \cdot S(n-1,k) + S(n-1,k-1)$}

\subsubsection{Bell Numbers}
\paragraph{Total Partitions: $B_n = \sum_{k=0}^{n} S(n,k)$}

\subsection{Asymptotic Methods in Combinatorics}
\label{subsec:asymptotic-combinatorics}

\subsubsection{Stirling's Approximation (Full Treatment)}
\paragraph{Statement: $n! \sim \sqrt{2\pi n} \left(\frac{n}{e}\right)^n$}
\paragraph{Proof via Euler-Maclaurin Formula.}
\paragraph{More Precise Expansions.}

\subsubsection{Asymptotic Analysis of Binomial Coefficients}
\paragraph{Central Binomial Coefficient: $\binom{2n}{n} \sim \frac{4^n}{\sqrt{\pi n}}$}

\subsubsection{Saddle-Point Method (Introduction)}

\subsection{Exercises}
\subsubsection{Warmup Problems (30 problems)}
\subsubsection{Standard Problems (40 problems)}
\subsubsection{Challenging Problems (30 problems)}
\subsubsection{Research Problems ($\star$) (15 problems)}

%----------------------------------------------
% SECTION 4: GRAPH THEORY
%----------------------------------------------
\section{Graph Theory Foundations}
\label{sec:graph-theory}

\begin{sectionintro}
Graphs are the most versatile mathematical structure in computer science. They 
model networks, dependencies, state spaces, data structures, and more. This 
section develops graph theory from first principles, emphasizing algorithmic 
applications.
\end{sectionintro}

\subsection{Basic Definitions and Representations}
\label{subsec:graph-basics}

\subsubsection{Graphs: Formal Definition}
\paragraph{Undirected Graphs: $G = (V, E)$}
\paragraph{Directed Graphs (Digraphs): $G = (V, A)$}
\paragraph{Multigraphs and Pseudographs.}

\subsubsection{Graph Terminology}
\paragraph{Vertices (Nodes) and Edges (Arcs).}
\paragraph{Degree: In-Degree, Out-Degree.}
\paragraph{Handshaking Lemma: $\sum_{v \in V} \deg(v) = 2|E|$}
\paragraph{Isolated, Pendant, and Universal Vertices.}

\subsubsection{Special Types of Graphs}
\paragraph{Complete Graphs: $K_n$}
\paragraph{Bipartite Graphs and Complete Bipartite Graphs: $K_{m,n}$}
\paragraph{Cycle Graphs: $C_n$}
\paragraph{Wheel Graphs: $W_n$}
\paragraph{Hypercubes: $Q_n$}

\subsubsection{Graph Representations}
\paragraph{Adjacency Matrix: $O(|V|^2)$ Space.}
\paragraph{Adjacency List: $O(|V| + |E|)$ Space.}
\paragraph{Edge List.}
\paragraph{Incidence Matrix.}
\paragraph{Trade-offs: Space vs. Query Time.}

\subsection{Connectivity and Paths}
\label{subsec:connectivity}

\subsubsection{Walks, Trails, Paths, and Cycles}
\paragraph{Definitions and Distinctions.}
\paragraph{Simple Paths vs. Paths with Repetition.}

\subsubsection{Connected Graphs}
\paragraph{Connected Components.}
\paragraph{Strong and Weak Connectivity in Digraphs.}

\subsubsection{Distance and Diameter}
\paragraph{Shortest Path Length: $d(u,v)$}
\paragraph{Diameter: $\max_{u,v} d(u,v)$}
\paragraph{Radius and Center.}

\subsubsection{Eulerian Paths and Circuits}
\paragraph{Euler's Theorem: Necessary and Sufficient Conditions.}
\paragraph{Fleury's Algorithm.}
\paragraph{Applications: DNA Sequencing, Route Planning.}

\subsubsection{Hamiltonian Paths and Circuits}
\paragraph{Definition and Examples.}
\paragraph{Dirac's and Ore's Theorems (Sufficient Conditions).}
\paragraph{NP-Completeness (Preview).}

\subsection{Trees and Forests}
\label{subsec:trees}

\subsubsection{Definition of Trees}
\paragraph{Acyclic Connected Graphs.}
\paragraph{Equivalent Characterizations of Trees.}

\subsubsection{Properties of Trees}
\paragraph{Theorem: A Tree with $n$ Vertices Has $n-1$ Edges.}
\paragraph{Leaves and Internal Nodes.}

\subsubsection{Rooted Trees}
\paragraph{Parent, Child, Sibling, Ancestor, Descendant.}
\paragraph{Height and Depth.}
\paragraph{Binary Trees, $k$-ary Trees.}

\subsubsection{Spanning Trees}
\paragraph{Definition and Existence.}
\paragraph{Number of Spanning Trees: Cayley's Formula.}
\paragraph{Minimum Spanning Tree Problem (Preview).}

\subsection{Directed Acyclic Graphs (DAGs)}
\label{subsec:dags}

\subsubsection{Definition and Properties}
\subsubsection{Topological Sorting}
\paragraph{Algorithm and Correctness.}
\paragraph{Applications: Task Scheduling, Precedence Constraints.}

\subsubsection{Transitive Closure}
\paragraph{Warshall's Algorithm.}

\subsection{Graph Coloring}
\label{subsec:coloring}

\subsubsection{Vertex Coloring}
\paragraph{Chromatic Number: $\chi(G)$}
\paragraph{Bounds: Brooks' Theorem.}

\subsubsection{Edge Coloring}
\paragraph{Chromatic Index: $\chi'(G)$}
\paragraph{Vizing's Theorem.}

\subsubsection{Applications}
\paragraph{Register Allocation in Compilers.}
\paragraph{Scheduling Problems.}

\subsection{Planar Graphs}
\label{subsec:planar}

\subsubsection{Definition and Examples}
\subsubsection{Euler's Formula: $V - E + F = 2$}
\subsubsection{Kuratowski's Theorem}
\subsubsection{Applications: Circuit Design, Geographic Maps.}

\subsection{Advanced Topics}
\label{subsec:graph-advanced}

\subsubsection{Matchings and Covers}
\paragraph{Maximum Matching.}
\paragraph{König's Theorem for Bipartite Graphs.}

\subsubsection{Network Flows (Introduction)}
\paragraph{Max-Flow Min-Cut Theorem (Statement).}

\subsubsection{Expander Graphs}
\paragraph{Applications to Derandomization.}

\subsection{Exercises}
\subsubsection{Warmup Problems (25 problems)}
\subsubsection{Standard Problems (35 problems)}
\subsubsection{Challenging Problems (25 problems)}
\subsubsection{Research Problems ($\star$) (10 problems)}

%----------------------------------------------
% SECTION 5: ALGEBRAIC STRUCTURES
%----------------------------------------------
\section{Algebraic Structures for Algorithm Design}
\label{sec:algebra}

\begin{sectionintro}
Abstract algebra provides powerful tools for algorithm design, especially in 
cryptography, coding theory, and symbolic computation. This section introduces 
groups, rings, and fields with algorithmic applications in mind.
\end{sectionintro}

\subsection{Groups}
\label{subsec:groups}

\subsubsection{Definition and Examples}
\paragraph{Axioms: Closure, Associativity, Identity, Inverses.}
\paragraph{Examples: $(\mathbb{Z}, +)$, $(\mathbb{Z}_n, +_n)$, $(S_n, \circ)$}

\subsubsection{Subgroups}
\paragraph{Definition and Lagrange's Theorem.}

\subsubsection{Cyclic Groups}
\paragraph{Generators and Order.}

\subsubsection{Permutation Groups}
\paragraph{Symmetric Group $S_n$.}
\paragraph{Cayley's Theorem.}

\subsubsection{Group Homomorphisms}
\paragraph{Kernels and Images.}

\subsection{Rings and Fields}
\label{subsec:rings-fields}

\subsubsection{Rings}
\paragraph{Definition and Examples: $\mathbb{Z}$, $\mathbb{Z}_n$, Polynomial Rings.}
\paragraph{Units and Zero Divisors.}
\paragraph{Integral Domains.}

\subsubsection{Fields}
\paragraph{Definition and Examples: $\mathbb{Q}$, $\mathbb{R}$, $\mathbb{Z}_p$}
\paragraph{Finite Fields (Galois Fields): $GF(p^n)$}

\subsubsection{Polynomial Arithmetic}
\paragraph{Division Algorithm for Polynomials.}
\paragraph{Greatest Common Divisor.}

\subsection{Applications to Cryptography}
\label{subsec:crypto-algebra}

\subsubsection{RSA and Group Theory}
\subsubsection{Diffie-Hellman Key Exchange}
\subsubsection{Elliptic Curve Cryptography (Overview)}

\subsection{Applications to Coding Theory}
\label{subsec:coding-algebra}

\subsubsection{Linear Codes}
\subsubsection{Cyclic Codes and BCH Codes}

\subsection{Exercises}
\subsubsection{Warmup Problems}
\subsubsection{Standard Problems}
\subsubsection{Challenging Problems}
\subsubsection{Research Problems ($\star$)}

%==============================================
% END OF CHAPTER 1
%==============================================
\chapter{Material Notation Systems}
\chapter{Agricultural Complexity and Token Systems}
\chapter{The Birth of Written Mathematics}