%----------------------------------------------
\part{Logic and the Foundations of Reasoning}
\label{part:logic-foundations}

\begin{partintro}
\lettrine[lines=3]{B}{efore mathematics}, there was logic. Before we can reason about numbers, structures, or algorithms, we must understand reasoning itself. This part develops the formal systems of propositional and predicate logic, proof theory, and the philosophical foundations that make mathematics possible.

\vspace{1em}
\textbf{What Makes This Different:}
\begin{itemize}[noitemsep]
    \item \textbf{Philosophical Depth:} From Aristotelian syllogisms to modern proof assistants
    \item \textbf{Complete Formalization:} Natural deduction, sequent calculus, resolution
    \item \textbf{Computational Connection:} Logic as the foundation of programming languages
    \item \textbf{Metamathematical Results:} Completeness, soundness, decidability
\end{itemize}

\begin{quote}
\textit{``Logic is the beginning of wisdom, not the end.''}

\hfill--- \textsc{Spock (and Aristotle, essentially)}
\end{quote}
\end{partintro}

\chapter{Propositional Logic and the Calculus of Reasoning}
\chapter{Predicate Logic and Quantificational Reasoning}
\chapter{Modal Logic: Necessity, Possibility, and Temporal Reasoning}
\chapter{Intuitionistic Logic and Constructive Mathematics}
\chapter{Mathematical Proof: Structure and Technique}
\chapter{Proof Theory and Natural Deduction}
\chapter{The Curry-Howard Correspondence: Proofs as Programs}
\chapter{Automated Theorem Proving and Proof Assistants}
\chapter{Metalogic: Completeness, Soundness, and Decidability}
\chapter{Gödel's Incompleteness Theorems: The Limits of Formal Systems}