%----------------------------------------------
% CHAPTER 1: Mathematical and Algorithmic Prerequisites
%----------------------------------------------
\chapter{Mathematical and Algorithmic Prerequisites}
\label{ch:prerequisites}

%===============================================
% CHAPTER INTRODUCTION
%===============================================
\lettrine[lines=3, lhang=0.1, loversize=0.15]{T}{his chapter} establishes 
the mathematical foundation required for rigorous algorithm analysis. We cover 
discrete mathematics, probability theory, mathematical analysis, linear algebra, 
and number theory—each selected for its direct relevance to algorithmic reasoning.

\begin{chapterintro}
\textbf{Why This Chapter Matters:}
\begin{itemize}[noitemsep]
    \item \textbf{Discrete Mathematics} provides the language of algorithms: sets, functions, graphs, and proof techniques
    \item \textbf{Probability Theory} enables average-case and randomized algorithm analysis
    \item \textbf{Mathematical Analysis} gives us tools for asymptotic reasoning and limits
    \item \textbf{Linear Algebra} underpins graph algorithms, Markov chains, and numerical methods
    \item \textbf{Number Theory} appears in cryptography, hashing, and randomization
\end{itemize}

\textbf{How to Use This Chapter:}
If you're already comfortable with a topic, skim it for notation and conventions.
Each section includes self-assessment problems to verify your understanding.
\end{chapterintro}

\section{Discrete Mathematics: The Language of Algorithms}
\label{sec:discrete-math}

\begin{sectionintro}
Discrete mathematics forms the grammatical structure of algorithm design.
Unlike continuous mathematics, we work with countable, distinct objects:
integers, graphs, logical propositions. This section covers the foundational
concepts that appear repeatedly throughout algorithm analysis.
\end{sectionintro}

\subsection{Sets, Functions, and Relations}
\label{subsec:sets-functions-relations}

\subsubsection{Set Theory Fundamentals}
\subsubsection{Operations on Sets: Union, Intersection, Difference, Cartesian Product}
\subsubsection{Functions: Injective, Surjective, Bijective Mappings}
\subsubsection{Relations and Equivalence Classes}
\subsubsection{Order Relations: Partial Orders, Total Orders, Lattices}

%-----------------------------------------------
\subsection{Combinatorics: Counting and Arrangements}
\label{subsec:combinatorics}

\begin{subsectionintro}
Combinatorics answers "how many?" questions essential to algorithm analysis:
How many possible inputs exist? How many comparisons are necessary?
\end{subsectionintro}

\subsubsection{Fundamental Counting Principles: Sum and Product Rules}
\subsubsection{Permutations: With and Without Repetition}
\subsubsection{Combinations and the Binomial Theorem}
\subsubsection{Pigeonhole Principle and Its Applications}
\subsubsection{Inclusion-Exclusion Principle}
\subsubsection{Generating Functions for Combinatorial Sequences}
\subsubsection{Combinatorial Identities and Proof Techniques}

\subsection{Graph Theory Essentials}
\label{subsec:graph-theory}

\begin{subsectionintro}
Graphs model relationships between entities. They appear everywhere:
networks, dependencies, state machines, data structures.
\end{subsectionintro}

\subsubsection{Graph Definitions: Vertices, Edges, Degree}
\subsubsection{Representations: Adjacency Matrix, Adjacency List, Edge List}
\subsubsection{Paths, Walks, Cycles, and Connectivity}
\subsubsection{Trees: Definitions, Properties, and Rooted Trees}
\subsubsection{Directed Acyclic Graphs (DAGs) and Topological Ordering}
\subsubsection{Special Graph Classes: Bipartite, Complete, Planar}
\subsubsection{Preview: BFS and DFS as Fundamental Traversals}

%-----------------------------------------------
\subsection{Proof Techniques: The Art of Rigorous Reasoning}
\label{subsec:proof-techniques}

\begin{subsectionintro}
Algorithm correctness and complexity bounds require proofs.
This subsection teaches you how to construct and verify arguments.
\end{subsectionintro}

\subsubsection{Direct Proof: Structure and Examples}
\subsubsection{Proof by Contradiction: Assuming the Negation}
\subsubsection{Proof by Contrapositive}
\subsubsection{Mathematical Induction: Weak Form}
\subsubsection{Strong Induction and the Well-Ordering Principle}
\subsubsection{Structural Induction for Recursive Definitions}
\subsubsection{Common Proof Pitfalls and How to Avoid Them}

%===============================================
% SECTION 2: PROBABILITY THEORY
%===============================================
\section{Probability Theory for Algorithm Analysis}
\label{sec:probability}

\begin{sectionintro}
Randomized algorithms and average-case analysis require probability.
This section builds intuition and formal tools for probabilistic reasoning.
\end{sectionintro}

%-----------------------------------------------
\subsection{Foundations: Sample Spaces and Events}
\label{subsec:sample-spaces}

\subsubsection{Sample Spaces: The Universe of Outcomes}
\subsubsection{Probability Axioms: Kolmogorov's Framework}
\subsubsection{Equally Likely Outcomes: The Discrete Uniform Distribution}
\subsubsection{Conditional Probability: Updating Beliefs}
\subsubsection{Law of Total Probability}
\subsubsection{Bayes' Theorem: Inverting Conditional Probabilities}


%-----------------------------------------------
\subsection{Random Variables and Expectations}
\label{subsec:random-variables}

\begin{subsectionintro}
Random variables map outcomes to numbers, allowing numerical analysis.
Expectation is the most important summary statistic in algorithm analysis.
\end{subsectionintro}
\subsubsection{Discrete Random Variables: Definition}
% $X: \Omega \to \mathbb{R}$

\subsubsection{Probability Mass Functions (PMF)}
% $p_X(x) = P(X = x)$

\subsubsection{Expected Value: Definition and Intuition}
% $E[X] = \sum_x x \cdot p_X(x)$

\subsubsection{Functions of Random Variables}
% $E[g(X)]$ computation

\subsubsection{Properties of Expectation}
% Linearity، monotonicity

%-----------------------------------------------
\subsection{Common Distributions in Algorithms}
\label{subsec:distributions}

\subsubsection{Bernoulli Distribution: Single Trial}

\subsubsection{Binomial Distribution: Multiple Independent Trials}

\subsubsection{Geometric Distribution: Waiting Times}

\subsubsection{Poisson Distribution: Rare Events}

\subsubsection{Uniform Distribution: Equally Likely Values}

%-----------------------------------------------
\subsection{Linearity of Expectation: The Most Powerful Tool}
\label{subsec:linearity-expectation}

\begin{subsectionintro}
Linearity of expectation works even when random variables are dependent—
this makes it extraordinarily useful in algorithm analysis.
\end{subsectionintro}

\subsubsection{Statement and Proof}

\subsubsection{Why Independence Is Not Required}

\subsubsection{Indicator Random Variables}

\subsubsection{Applications in Algorithm Analysis}

\subsubsection{Case Study: Expected Number of Comparisons in QuickSort}

%-----------------------------------------------
\subsection{Independence and Conditional Expectation}
\label{subsec:independence}

\subsubsection{Independence of Events}
\subsubsection{Independence of Random Variables}
\subsubsection{Conditional Expectation: $E[X | Y]$}
\subsubsection{Applications to Randomized Algorithms}

%-----------------------------------------------
\subsection{Variance and Concentration}
\label{subsec:variance}

\subsubsection{Variance: Measuring Spread}
\subsubsection{Standard Deviation}
\subsubsection{Properties of Variance}
\subsubsection{Chebyshev's Inequality: Probabilistic Bounds}
\subsubsection{Preview: Chernoff Bounds (Covered in Part 2)}

%-----------------------------------------------
\subsection{Moment Generating Functions (Brief Preview)}
\label{subsec:mgf}

\subsubsection{Definition and Intuition}
\subsubsection{Using MGFs to Compute Moments}
\subsubsection{Sum of Independent Random Variables}


%===============================================
% SECTION 3: MATHEMATICAL ANALYSIS
%===============================================
\section{Mathematical Analysis: Limits and Asymptotics}
\label{sec:analysis}

\begin{sectionintro}
Algorithm analysis is fundamentally about asymptotic behavior as input size grows.
This section provides the calculus and analysis tools needed for rigorous reasoning.
\end{sectionintro}

%-----------------------------------------------
\subsection{Limits and Continuity}
\label{subsec:limits}

\subsubsection{Limits of Sequences: $\lim_{n \to \infty} a_n$}
\subsubsection{Limit Laws: Addition, Multiplication, Composition}
\subsubsection{L'Hôpital's Rule: Resolving Indeterminate Forms}
\subsubsection{Asymptotic Equivalence: $f \sim g$}
\subsubsection{Landau Notation: Informal Preview}

%-----------------------------------------------
\subsection{Sequences and Series}
\label{subsec:sequences-series}

\subsubsection{Convergence of Sequences}
\subsubsection{Infinite Series: $\sum_{n=1}^{\infty} a_n$}
\subsubsection{Geometric Series: $\sum_{k=0}^{n} r^k$}
\subsubsection{Harmonic Series and Generalizations}
\subsubsection{Power Series and Radius of Convergence}
%-----------------------------------------------
\subsection{Summations and Closed Forms}
\label{subsec:summations}

\begin{subsectionintro}
Converting sums to closed forms is essential for exact complexity analysis.
\end{subsectionintro}

\subsubsection{Basic Summation Formulas}
\subsubsection{Arithmetic and Geometric Progressions}
\subsubsection{Techniques for Finding Closed Forms}
\subsubsection{Perturbation Method}
\subsubsection{Repertoire Method (Concrete Mathematics Approach)}
\subsubsection{Euler-Maclaurin Formula: Sum-Integral Connection}

%-----------------------------------------------
\subsection{Integration and Differentiation}
\label{subsec:calculus}

\subsubsection{Fundamental Theorem of Calculus}
\subsubsection{Integration Techniques: Substitution, Parts, Partial Fractions}
\subsubsection{Approximating Sums with Integrals}
\subsubsection{Growth Rates via Derivatives}

%-----------------------------------------------
\subsection{Taylor Series and Asymptotic Expansions}
\label{subsec:taylor-series}

\subsubsection{Taylor Series: Definition and Motivation}
\subsubsection{Common Taylor Series}
\subsubsection{Using Taylor Series for Approximation}
\subsubsection{Asymptotic Expansions}

%-----------------------------------------------
\subsection{Stirling's Approximation}
\label{subsec:stirling}

\begin{subsectionintro}
Factorials grow faster than polynomials but slower than exponentials.
Stirling's formula provides precise asymptotic behavior.
\end{subsectionintro}

\subsubsection{Statement of Stirling's Formula}
\subsubsection{Proof Sketch via Integral Approximation}
\subsubsection{Applications in Combinatorics}
\subsubsection{More Precise Versions and Error Bounds}

%===============================================
% SECTION 4: LINEAR ALGEBRA
%===============================================
\section{Linear Algebra for Algorithmic Applications}
\label{sec:linear-algebra}

\begin{sectionintro}
Linear algebra appears in graph algorithms, Markov chains, numerical methods,
and machine learning. We focus on computational and algorithmic aspects.
\end{sectionintro}

%-----------------------------------------------
\subsection{Vectors, Matrices, and Linear Systems}
\label{subsec:vectors-matrices}

\subsubsection{Vector Spaces and Subspaces}
\subsubsection{Linear Independence, Span, and Basis}
\subsubsection{Matrix Operations: Addition, Multiplication, Transpose}
\subsubsection{Matrix Multiplication as Linear Transformation Composition}
\subsubsection{Systems of Linear Equations: Gaussian Elimination}

%-----------------------------------------------
\subsection{Eigenvalues and Eigenvectors}
\label{subsec:eigenvalues}

\subsubsection{Definitions: $Av = \lambda v$}
\subsubsection{Characteristic Polynomial: $\det(A - \lambda I) = 0$}
\subsubsection{Diagonalization: $A = PDP^{-1}$}
\subsubsection{Spectral Theorem for Symmetric Matrices}
\subsubsection{Applications to Algorithm Analysis}

%-----------------------------------------------
\subsection{Markov Chains and Random Walks}
\label{subsec:markov-chains}

\begin{subsectionintro}
Markov chains model random processes with memoryless transitions.
They appear in randomized algorithms and probabilistic analysis.
\end{subsectionintro}

\subsubsection{Stochastic Matrices: Definition and Properties}
\subsubsection{Steady-State Distribution: $\pi P = \pi$}
\subsubsection{Random Walks on Graphs}
\subsubsection{Case Study: PageRank Algorithm}

%-----------------------------------------------
\subsection{Matrix Operations and Complexity}
\label{subsec:matrix-complexity}

\subsubsection{Standard Matrix Multiplication: $O(n^3)$}
\subsubsection{Strassen's Algorithm: $O(n^{\log_2 7}) \approx O(n^{2.81})$}
\subsubsection{Coppersmith-Winograd and Beyond}
\subsubsection{Matrix Inversion Complexity}
\subsubsection{Solving Linear Systems: LU Decomposition, Cholesky}

%===============================================
% SECTION 5: NUMBER THEORY
%===============================================
\section{Number Theory Essentials}
\label{sec:number-theory}

\begin{sectionintro}
Number theory provides tools for cryptography, hashing, pseudorandom generation,
and modular arithmetic—all crucial in modern algorithms.
\end{sectionintro}

%-----------------------------------------------
\subsection{Divisibility and Modular Arithmetic}
\label{subsec:divisibility}

\subsubsection{Division Algorithm: $a = bq + r$}
\subsubsection{Modular Arithmetic: $a \equiv b \pmod{m}$}
\subsubsection{Properties of Modular Operations}
\subsubsection{Modular Inverses: $ax \equiv 1 \pmod{m}$}
\subsubsection{Chinese Remainder Theorem}

%-----------------------------------------------
\subsection{Prime Numbers and Factorization}
\label{subsec:primes}

\subsubsection{Definition of Prime Numbers}
\subsubsection{Fundamental Theorem of Arithmetic}
\subsubsection{Prime Distribution: Prime Number Theorem}
\subsubsection{Sieve of Eratosthenes: $O(n \log \log n)$}
\subsubsection{Primality Testing: Fermat, Miller-Rabin}

%-----------------------------------------------
\subsection{Greatest Common Divisor and Euclidean Algorithm}
\label{subsec:gcd}

\begin{subsectionintro}
The Euclidean algorithm is one of the oldest and most elegant algorithms.
Its analysis demonstrates beautiful connections between number theory and complexity.
\end{subsectionintro}

\subsubsection{GCD Definition and Properties}
\subsubsection{Euclidean Algorithm}
\subsubsection{Extended Euclidean Algorithm}
\subsubsection{Complexity Analysis: $O(\log \min(a, b))$}
\subsubsection{Lamé's Theorem: Connection to Fibonacci Numbers}

%-----------------------------------------------
\subsection{Applications to Cryptography and Hashing}
\label{subsec:crypto-hash}

\subsubsection{Modular Exponentiation: Fast Algorithm}
\subsubsection{RSA Cryptosystem: Overview}
\subsubsection{Hash Functions: Collision Resistance}
\subsubsection{Universal Hashing (Preview)}

%===============================================
% SECTION 6: ADDITIONAL TOOLS
%===============================================
\section{Additional Mathematical Tools}
\label{sec:additional-tools}

%-----------------------------------------------
\subsection{Logarithms and Exponentials}
\label{subsec:log-exp}

\subsubsection{Properties of Logarithms}
\subsubsection{Change of Base Formula}
\subsubsection{Natural vs. Binary Logarithms in CS}
\subsubsection{Exponential Growth and Decay}
\subsubsection{Comparing Growth Rates}

%-----------------------------------------------
\subsection{Floor and Ceiling Functions}
\label{subsec:floor-ceiling}

\subsubsection{Definitions: $\lfloor x \rfloor$ and $\lceil x \rceil$}
\subsubsection{Useful Identities and Properties}
\subsubsection{Applications in Algorithm Analysis}
%-----------------------------------------------
\subsection{Asymptotic Notation: Informal Preview}
\label{subsec:asymptotic-preview}

\begin{subsectionintro}
We provide intuition here; formal definitions appear in Part 2, Chapter 2.
\end{subsectionintro}

\subsubsection{Intuition Behind $O$, $\Omega$, $\Theta$}
\subsubsection{Why We Need Formal Definitions}
\subsubsection{Common Growth Rate Hierarchy}

%===============================================
% SECTION 7: SELF-ASSESSMENT AND RESOURCES
%===============================================
\section{Self-Assessment and Further Study}
\label{sec:self-assessment}

\begin{sectionintro}
Test your understanding with these exercises before proceeding to Part 2.
If you struggle with a section, revisit the material or consult the recommended resources.
\end{sectionintro}

%-----------------------------------------------
\subsection{Self-Assessment Exercises}
\label{subsec:exercises}

\subsubsection{Discrete Mathematics Problems}
\subsubsection{Probability Problems}
\subsubsection{Analysis Problems}
\subsubsection{Linear Algebra Problems}
\subsubsection{Number Theory Problems}

%-----------------------------------------------
\subsection{Further Reading and Resources}
\label{subsec:resources}

\subsubsection{Recommended Textbooks}
\subsubsection{Online Resources and Video Lectures}
\subsubsection{Practice Problem Collections}

%-----------------------------------------------
\subsection{Notation and Conventions Used in This Book}
\label{subsec:notation}

\subsubsection{Standard Mathematical Symbols}
\subsubsection{Asymptotic Notation}
\subsubsection{Probability Notation}
\subsubsection{Graph Notation}