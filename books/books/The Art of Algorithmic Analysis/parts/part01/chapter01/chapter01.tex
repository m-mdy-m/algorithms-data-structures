\chapter{Why ``Precise Analysis'' Matters — From Theory to Engineering}

\section{The Illusion of Intuitive Understanding}
\subsection{Case Study 1: When $O(n^2)$ Beats $O(n \log n)$ in Practice}
\subsection{Case Study 2: The QuickSort Paradox — $O(n^2)$ Worst-Case, Yet Industry Standard}
\subsection{Case Study 3: Hash Tables vs. Binary Search Trees — Theory vs. Reality}

\section{The Gap Between Theoretical Complexity and Real-World Performance}
\subsection{Hidden Constants in Big-O Notation}
\subsection{Lower-Order Terms That Dominate at Practical Scales}
\subsection{Memory Hierarchy Effects Invisible to RAM Model}
\subsection{Architecture-Specific Considerations}

\section{The Role of Constants, Lower-Order Terms, and Hardware}
\subsection{Quantifying Constants: From Theory to Measurement}
\subsection{When Lower-Order Terms Matter}
\subsection{Cache Effects and Memory Bandwidth}
\subsection{Branch Prediction and Pipeline Stalls}
\subsection{SIMD and Vectorization Opportunities}

\section{Bridging the Theory-Practice Gap}
\subsection{Algorithm Engineering Principles}
\subsection{When to Trust Theory}
\subsection{When to Question Theory}
\subsection{Hybrid Approaches: Combining Analytical and Empirical Methods}

\section{Why We Still Need Asymptotic Analysis}
\subsection{Scalability Prediction}
\subsection{Algorithm Comparison Across Architectures}
\subsection{Identifying Bottlenecks}
\subsection{Guiding Optimization Efforts}

\section{Exercises}