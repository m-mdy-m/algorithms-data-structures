
%----------------------------------------------
% CHAPTER 1: Mathematical and Algorithmic Prerequisites
%----------------------------------------------
\chapter{Mathematical and Algorithmic Prerequisites}
\label{ch:prerequisites}

%===============================================
% CHAPTER INTRODUCTION
%===============================================
\lettrine[lines=3, lhang=0.1, loversize=0.15]{T}{his chapter} establishes 
the mathematical foundation required for rigorous algorithm analysis. We cover 
discrete mathematics, probability theory, mathematical analysis, linear algebra, 
and number theory—each selected for its direct relevance to algorithmic reasoning.

\begin{chapterintro}
\textbf{Why This Chapter Matters:}
\begin{itemize}[noitemsep]
    \item \textbf{Discrete Mathematics} provides the language of algorithms: sets, functions, graphs, and proof techniques
    \item \textbf{Probability Theory} enables average-case and randomized algorithm analysis
    \item \textbf{Mathematical Analysis} gives us tools for asymptotic reasoning and limits
    \item \textbf{Linear Algebra} underpins graph algorithms, Markov chains, and numerical methods
    \item \textbf{Number Theory} appears in cryptography, hashing, and randomization
\end{itemize}

\textbf{How to Use This Chapter:}
If you're already comfortable with a topic, skim it for notation and conventions.
Each section includes self-assessment problems to verify your understanding.
\end{chapterintro}

\section{Discrete Mathematics: The Language of Algorithms}
\label{sec:discrete-math}

\begin{sectionintro}
Discrete mathematics forms the grammatical structure of algorithm design.
Unlike continuous mathematics, we work with countable, distinct objects:
integers, graphs, logical propositions. This section covers the foundational
concepts that appear repeatedly throughout algorithm analysis.
\end{sectionintro}
\subsection{Sets, Functions, and Relations}
\subsubsection{Set Theory Basics}
\subsubsection{Operations on Sets}
\subsubsection{Functions: Injective, Surjective, Bijective}
\subsubsection{Relations and Equivalence Classes}
\subsubsection{Order Relations and Partial Orders}

\subsection{Combinatorics: Permutations, Combinations, and Binomial Coefficients}
\subsubsection{The Fundamental Counting Principles}
\subsubsection{Permutations with and without Repetition}
\subsubsection{Combinations and the Binomial Theorem}
\subsubsection{Pigeonhole Principle and Applications}
\subsubsection{Inclusion-Exclusion Principle}
\subsubsection{Combinatorial Identities Useful in Analysis}

\subsection{Graph Theory Basics}
\subsubsection{Graph Definitions and Representations}
\subsubsection{Paths, Cycles, and Connectivity}
\subsubsection{Trees and Their Properties}
\subsubsection{Directed Graphs and DAGs}
\subsubsection{Graph Algorithms Preview (BFS, DFS)}

\subsection{Proof Techniques: Induction, Contradiction, and Contrapositive}
\subsubsection{Direct Proof Structure}
\subsubsection{Proof by Contradiction}
\subsubsection{Proof by Contrapositive}
\subsubsection{Mathematical Induction: Weak Form}
\subsubsection{Strong Induction and Well-Ordering Principle}
\subsubsection{Structural Induction for Recursive Definitions}
\subsubsection{Common Proof Pitfalls and How to Avoid Them}

\section{Elementary Probability Theory}
\subsection{Sample Spaces, Events, and Probability Measures}
\subsubsection{Sample Spaces and Events}
\subsubsection{Probability Axioms and Basic Properties}
\subsubsection{Equally Likely Outcomes}
\subsubsection{Conditional Probability}

\subsection{Random Variables and Expectations}
\subsubsection{Discrete Random Variables}
\subsubsection{Probability Mass Functions}
\subsubsection{Expected Value: Definition and Properties}
\subsubsection{Functions of Random Variables}

\subsection{Basic Distributions: Uniform, Bernoulli, Geometric, Binomial}
\subsubsection{Uniform Distribution}
\subsubsection{Bernoulli Trials}
\subsubsection{Binomial Distribution}
\subsubsection{Geometric Distribution}
\subsubsection{Negative Binomial Distribution}

\subsection{Linearity of Expectation}
\subsubsection{Statement and Proof}
\subsubsection{Why Independence Is Not Required}
\subsubsection{Applications in Algorithm Analysis}
\subsubsection{Example: Expected Comparisons in QuickSort}

\subsection{Conditional Probability and Independence}
\subsubsection{Conditional Probability Revisited}
\subsubsection{Law of Total Probability}
\subsubsection{Bayes' Theorem}
\subsubsection{Independence of Events}
\subsubsection{Independence of Random Variables}
\subsubsection{Conditional Expectation}

\subsection{Variance and Standard Deviation}
\subsubsection{Definitions and Properties}
\subsubsection{Variance of Sums}
\subsubsection{Chebyshev's Inequality (Preview)}

\subsection{Moment Generating Functions (Brief Introduction)}
\subsubsection{Definition and Basic Properties}
\subsubsection{Using MGFs to Find Moments}
\subsubsection{MGFs and Sum of Independent Random Variables}

\section{Mathematical Analysis}
\subsection{Limits, Continuity, and Asymptotic Behavior}
\subsubsection{Limits of Sequences}
\subsubsection{Limit Laws}
\subsubsection{L'Hôpital's Rule}
\subsubsection{Asymptotic Equivalence}
\subsubsection{Landau Notation (Informal Preview)}

\subsection{Sequences and Series}
\subsubsection{Convergence of Sequences}
\subsubsection{Infinite Series: Convergence Tests}
\subsubsection{Geometric Series}
\subsubsection{Harmonic Series and Generalizations}
\subsubsection{Power Series}

\subsection{Summations and Closed Forms}
\subsubsection{Basic Summation Formulas}
\subsubsection{Arithmetic and Geometric Sums}
\subsubsection{Techniques for Finding Closed Forms}
\subsubsection{Perturbation Method}
\subsubsection{Repertoire Method}
\subsubsection{Euler-Maclaurin Formula (Overview)}

\subsection{Integration and Differentiation (Brief Review)}
\subsubsection{Fundamental Theorem of Calculus}
\subsubsection{Integration Techniques Relevant to Analysis}
\subsubsection{Approximating Sums with Integrals}
\subsubsection{Growth Rates via Derivatives}

\subsection{Taylor Series and Asymptotic Expansions}
\subsubsection{Taylor Series Definition}
\subsubsection{Common Taylor Series}
\subsubsection{Using Taylor Series for Approximation}
\subsubsection{Asymptotic Expansions}

\subsection{Stirling's Approximation}
\subsubsection{Statement of Stirling's Formula}
\subsubsection{Proof Sketch}
\subsubsection{Applications in Combinatorics and Analysis}
\subsubsection{More Precise Versions}

\section{Linear Algebra (Brief Overview)}
\subsection{Vectors, Matrices, and Linear Transformations}
\subsubsection{Vector Spaces and Subspaces}
\subsubsection{Linear Independence and Basis}
\subsubsection{Matrix Operations}
\subsubsection{Matrix Multiplication as Composition}
\subsubsection{Systems of Linear Equations}

\subsection{Eigenvalues and Eigenvectors}
\subsubsection{Definitions}
\subsubsection{Characteristic Polynomial}
\subsubsection{Diagonalization}
\subsubsection{Spectral Theorem (for Symmetric Matrices)}

\subsection{Applications to Markov Chains and Graph Algorithms}
\subsubsection{Stochastic Matrices}
\subsubsection{Steady-State Distribution}
\subsubsection{Random Walks on Graphs}
\subsubsection{PageRank Algorithm (Overview)}

\subsection{Matrix Operations and Complexity}
\subsubsection{Standard Matrix Multiplication: $O(n^3)$}
\subsubsection{Strassen's Algorithm: $O(n^{\log_2 7})$}
\subsubsection{Matrix Inversion Complexity}
\subsubsection{Solving Linear Systems}

\section{Number Theory Essentials}
\subsection{Divisibility and Modular Arithmetic}
\subsubsection{Division Algorithm}
\subsubsection{Modular Arithmetic Operations}
\subsubsection{Congruences and Properties}
\subsubsection{Modular Inverses}

\subsection{Prime Numbers and Factorization}
\subsubsection{Prime Number Definition}
\subsubsection{Fundamental Theorem of Arithmetic}
\subsubsection{Prime Distribution and the Prime Number Theorem}
\subsubsection{Sieve of Eratosthenes}

\subsection{Greatest Common Divisor and Euclidean Algorithm}
\subsubsection{GCD Definition and Properties}
\subsubsection{Euclidean Algorithm}
\subsubsection{Extended Euclidean Algorithm}
\subsubsection{Complexity Analysis of Euclidean Algorithm}
\subsubsection{Lamé's Theorem}

\subsection{Applications to Cryptography and Hashing}
\subsubsection{Modular Exponentiation}
\subsubsection{RSA Cryptosystem (Overview)}
\subsubsection{Hash Functions and Collision Resistance}
\subsubsection{Universal Hashing (Preview)}

\section{Additional Mathematical Tools}
\subsection{Logarithms and Exponentials}
\subsubsection{Properties of Logarithms}
\subsubsection{Change of Base Formula}
\subsubsection{Natural Logarithm vs. Binary Logarithm}
\subsubsection{Exponential Growth and Decay}
\subsubsection{Comparing Logarithmic and Polynomial Growth}

\subsection{Floor and Ceiling Functions}
\subsubsection{Definitions and Basic Properties}
\subsubsection{Useful Identities}
\subsubsection{Applications in Algorithm Analysis}

\subsection{Asymptotic Notation (Informal Preview)}
\subsubsection{Intuition Behind $O$, $\Omega$, $\Theta$}
\subsubsection{Why We Need Formal Definitions (Chapter 7)}
\subsubsection{Common Growth Rates}

\section{Self-Assessment Exercises}
\subsection{Discrete Mathematics Problems}
\subsection{Probability Problems}
\subsection{Analysis Problems}
\subsection{Linear Algebra Problems}
\subsection{Number Theory Problems}

\section{Further Reading and Resources}
\subsection{Recommended Textbooks for Each Topic}
\subsection{Online Resources and Video Lectures}
\subsection{Practice Problem Collections}
